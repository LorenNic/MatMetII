%!TEX root = ../MatMetII.tex
\chapter{Acciai per impieghi strutturali}\label{chp:AcciaiStrutturali}
Tra gli accia per impieghi strutturali, si possono trovare sicuramente gli acciai per uso comune e gli accia per costruzioni speciali: ciò per via della grande varietà di prodotti che si possono produrre in questo ambito.
Giusto per avere un'idea di massima: gli acciai per uso comune ricoprono circa l'80\% della produzione per questa categoria.
Parliamo di accia che sono designasti, in generale, tramite la lettera '\textbf{S}' secondo la normativa \ref{sc:10027-1}.

In generale sono forniti come prodotti piani e lunghi. Possono uscire in diverse forme di finitura:
\begin{itemize}
\item allo stato di lavorazione a caldo;
\item allo stato normalizzato o bonificato;
\item ecc\dots
\end{itemize}
I prodotti sono normati dalla UNI EN 10149 che è la norma prodotto di riferimento.

Come accennato esiste una normativa sulla definizione dei prodotti in acciai, distinguendo tra
\begin{itemize}
\item Prodotti piani:
	\begin{itemize}
	\item larghi piatti,
	\item lamiere,
	\item nastri,
	\item lamiere profilate (nervate, ondulate)
	\end{itemize}
\item Prodotti lunghi:
	\begin{itemize}
	\item verghelle,
	\item filo,
	\item barre,
	\item ecc\dots
	\end{itemize}
\end{itemize}

La composizione chimica si riferisce all'analisi di colata, se non diversamente specificato dalla normativa.
In generale sono descritte le $wt.\%$ massime dei vari elementi in lega, tra cui anche il carbonio, salvo specificarne diversa presenza si una piccola quantità di qualche elemento.
La norma, di solito, specifica il raggiungimento di alcune proprietà meccaniche tra cui: valori minimi di $R_s$ o $R_m$ ed eventuali caratteristiche utili al fine della costruzione come la saldabilità.
Quando è prevista zincatura per immissione a caldo di un acciaio, deve esserne garantita l'idoneità: in genere tutti gli acciai possono subire questo tipo di finitura superficiale. Però acciai adatti riescono a formare delle fasi di precipitato di zinco che garantiscono resistenza maggiorata alla corrosione ambientale. Altri, non particolarmente adatti a tale trattamento, tendono a formare delle fasi di precipitato molto irregolari e grossolane che limitano, o addirittura peggiorano, la resistenza alla corrosione.

Tipicamente, la zincabilità dipende dal contenuto in lega del Si.
Allora si possono avere diverse situazioni:
\begin{description}
\item[$Si<0.03\%$] si è in una situazione cautelativa, sicuramente si ha una buona zincatura.
\item[$0.03\%<Si<0.12\%$] È comunque zincabile, ma non ha le stesse caratteristiche di resistenza alla corrosione del primo caso. Si definiscono, dunque, delle classi di zincabilità.
\item[$Si>0.3\%$] La sequenza delle fasi di zincatura non è garantita, dunque anche la resistenza all'atmosfera non è garantita. Può migliorare la resistenza alla corrosione in maniera marginale.
\end{description}
Tra l'altro è opportuno ricordare che tale trattamento tende ad infragilire il materiale. Fenomeno esaltato dal invecchiamento.

Inoltre sono riportati all'appendice \ref{sc:AccEffCalm} la definizione degli acciai calmati ed effervescenti.

Come già accennato per questa tipologia di acciai: spesso è richiesto il soddisfacimento del requisito di saldabilità.
Viene definito un acciaio saldabile se: 

\begin{definition}{Saldabilità}{saldabilità}
Un acciaio può essere considerato saldabile se può essere sottoposto a tale processo costruttivo con le normali tecniche di cantiere senza necessità di trattamenti temici post-saldatura. 
\end{definition}

Inoltre, vale la pena ricordare che più un acciaio è temprabile, più questo sarà meno saldabile.
Questo perché un acciaio fortemente temprabile forma più facilmente strutture rigide ma fragili. Dunque un processo di saldatura, dal punto di vista del materiale, può essere considerato come una tempra con raffreddamento in aria.
In genere viene definito un parametro di carbonio equivalente detto $CEV$.
Si considera, con opportune eccezioni, saldabile un acciaio con $CEV < 0.5$.
Questo parametro non è risolutivo: qualsiasi acciaio si può saldare. Aumenta la probabilità di formare delle strutture fragili nella zona termicamente alterata $ZTA$.
In questi casi bisogna ricorrere a tecniche di saldatura più avanzate.

\section{UNI EN 10025-(3-6) Prodotti laminati a caldo}
Riprendendo la normativa UNI EN 10025 già citata al capitolo \ref{sc:10027-1}.
Si ricorda che tale normativa è dedicata a \emph{prodotti laminati a caldo per impieghi strutturali} e che è divisa in sei parti.

\begin{enumerate}
\item Condizioni tecniche generali di fornitura
\item Condizioni tecniche di fornitura di acciai non legati per impieghi strutturali
\item Condizioni tecniche di fornitura di acciai per impieghi strutturali saldabili a grano fine allo stato normalizzato/normalizzato laminato
\item Condizioni tecniche di fornitura di acciai per impieghi strutturali saldabili a grano fine ottenuti mediante laminazione termomeccanica
\item Condizioni tecniche di fornitura di acciai per impieghi strutturali con resistenza migliorata alla corrosione atmosferica
\item Condizioni tecniche di fornitura per prodotti piani di acciai per impieghi strutturali ad alto limite di snervamento allo stato bonificato
\end{enumerate}

Sono compresi nella normativa acciai di tipo \textbf{S} garantendone il valore minimo di snervamento garantito e l'indice di resilienza come indicato dalla \ref{sc:10027-1}.
In più la norma definisce ulteriori sigle per indicare l'appartenenza di tali acciai ad una ben specifica parte della normativa:
\begin{description}
\item[+AR] indica \eng{As Rolled} ovvero acciaio grezzo da laminazione;
\item[+N] acciaio proveniente da laminazione normalizzata;
\item[+M] acciaio proveniente da laminazione temomeccanica;
\item[+W] acciaio a migliorata resistenza atmosferica;
\item[+Q] acciaio ad alto valore di snervamento allo stato bonificato.
\end{description}

\begin{example}{Esempio}
Acciaio UNI EN 10025-2: \texttt{S235J0C+N} ovvero:
\begin{description}
\item[S] acciaio per impieghi strutturali
\item[235] resistenza allo snervamento minima garantita in $\unit{\MPa}$
\item[J0] resilienza garantita maggiore di $27\unit{\J}$ ad una temperatura di $0\unit{\celsius}$
\item[C] acciaio adatto per la formatura a freddo
\item[N] acciaio allo stato normalizzato
\end{description}
\end{example}

La norma definisce anche quali siano le informazioni che devono essere cedute al committente:
\begin{itemize}
\item quantitativo da fornire;
\item forma del prodotto e numero della norma per dimensioni e tolleranze;
\item Dimensioni nominali e tolleranze dimensionali di forma;
\item Designazione dell'acciaio;
\item Tipi di documenti di controllo;
\item Requisiti aggiuntivi di controllo e prova e tutte le operazioni richieste.
\end{itemize}

\begin{example}{Esempi designazione della norma}
\begin{itemize}
\item Acciaio \texttt{UNI EN 10025-3} - \texttt{S275N} o \texttt{S275NL}
\item Acciaio \texttt{UNI EN 10025-4} - \texttt{S420M} o \texttt{S420ML}
\item Acciaio \texttt{UNI EN 10025-5} - \texttt{S355J0W+N} o \texttt{S355J0WP+N}
\item Acciaio \texttt{UNI EN 10025-6} - \texttt{S460Q} o \texttt{S460QL} o \texttt{S460QL1}
\end{itemize}
\end{example}
L'indice \texttt{L} indica valore di resilienza minima garantita più alta rispetto al solo parametro \texttt{N} a diverse temperature.
Stesso discorso per i parametri \texttt{M} e \texttt{ML} Solo che la condizione finale di fornitura è diversa.
Mentre per gli acciai che presentano il temine \texttt{W} o \texttt{WP} hanno dei tenori di Cr e Cu ben specificati dalla normativa. Quindi indicano una diversa composizione di lega.

\section{Acciai resistenti alla corrosione atmosferica}
Sono commercialmente definiti come acciai \textbf{Cor-Ten} che sta ad indicare:
\begin{description}
\item[\eng{Cor}] "\eng{Corrosion resistance}"
\item[\eng{Ten}] "\eng{Tensile strength}"
\end{description}

Per aumentare la resistenza atmosferica contengono come elementi in lega:
Cu, Cr, Ni e in caso tenori variabili di P.
La composizione tipica potrebbe essere:
\begin{example}{Composizione tipica CORTEN}
\begin{tabularx}{\textwidth}{XXXXXXX}
\toprule
C & Mn & Si & Cr & Ni & Cu & V\\
0.15 & 1.10 & 0.80 & 0.50 & 0.70 & 0.30 & 0.05\\
\bottomrule
\end{tabularx}
\end{example}
In genere questi acciai vengono forniti allo stato di laminazione di
normalizzazione o semplicemente laminato. Oppure sotto forma di barre, 
profilati e lamiere.
Altre normative descrivono questi acciai come adatti per applicazioni architettoniche ed eventualmente adatti per applicazioni più sollecitate.

Sono acciai che generalmente sono saldabili, eventualmente il materiale
di apporto che vengono usati anche per acciai Cr-Mn. Nel caso sia
necessaria una certa finitura estetica (caso di applicazioni
architettoniche) si usano elettrodi contenenti del Ni ($\approx 2 \div 
3\%$) per ottenere la stessa finitura superficiale.

\section{Acciai ad alta resistenza (HSS e AHSS)}
Sono acciai che hanno un preciso scopo: garantire alte prestazioni 
meccaniche senza eccedere con elementi leganti per mantenere un costo 
del materiale alto.
In particolare, le caratteristiche da ricercare in questo acciaio sono:
\begin{itemize}
\item Carico di rottura
\item Carico di snervamento
\item tenacità
\item talvolta una migliorata resistenza alla corrosione 
atmosferica e alle atmosfere industriali (potenzialmente 
aggressive)
\end{itemize}
Lo sviluppo degli acciai \textit{micro-legati} ha consentito
alla realizzazione di dimensionamenti a minore quantità di materiale
senza perdere di tenacità e duttilità ed altre caratteristiche
meccaniche.
Inoltre la micro-legatura ha tolto la necessità di
eseguire dei trattamenti termici successivi. Che ovviamente
si traduce per entrambi gli scopi in minori costi di produzione.

\subsection{Meccanismi di rinforzo}
Per raggiungere gli obbiettivi indicati in precedenza, gli acciai 
ad alta resistenza presentano i così detti \emph{Meccanismi di rinforzo}.
Si vedranno ora i principali.

\paragraph{Rafforzamento per soluzione solida}
Si tenda ad aumentare il tenore di carbonio, così il materiale diventa
più duro. In più, si sostituiscono gli elementi sostituzionali: tali 
deformano il reticolo cristallino interagendo con le dislocazioni, di 
fatto bloccandole.

\paragraph{Incrudimento}
La semplice lavorazione a freddo porta il materiale ad incrudire, 
per cui risulta più duro. Ciò è dovuto all'aumento delle dislocazioni.

\paragraph{Rinforzo per precipitazione}
Vengo costituite, tramite opportuni processi produttivi, delle fasi
solide sovrassature che pongono un ostacolo al movimento delle
dislocazioni.

I metodi presentati in precedenza, sebbene migliorino la resistenza
in generale, costituiscono un problema in termini di duttilità:
Andando a bloccare le dislocazioni si ha effettivamente un materiale
più duro, al contempo si perde di duttilità.
Alla figura \ref{fig:RinfTenacita} sono riportati qualitativamente.

\begin{figure}
\centering
\subfloat[][\emph{Effetti negativi}\label{fig:EffettiNegativiRinforzo}]
{\includegraphics[width = 0.4\textwidth]{EffettoRinforzoTenacita}} \quad
\subfloat[][\emph{Effetti positivi}\label{fig:EffettiNegativiRinforzo}]
{\includegraphics[width = 0.4\textwidth]{EffettoRinforzoTenacita2}}
\caption{meccanismi di rinforzo in rapporto alla tenacità}
\label{fig:RinfTenacita}
\end{figure}

\paragraph{Rafforzamento per affinamento del grano}
Nel caso di strutture cristalline CCC, come per acciai 
ferritico-perlitici per costruzione saldate, è l'unico meccanismo 
che incrementa entrambe le qualità: resistenza e tenacità.
Allora valgono:
\begin{align}
R_s &= \sigma_0 + K d^{-1/2} \label{eqn:tensServ}\\
I.T.T. &= A - B \ln d^{-1/2} \label{eqn:ITT}
\end{align}
Dalla \eqref{eqn:tensServ} e \eqref{eqn:ITT} è evidente come le 
dimensioni dimensioni del grano siano fondamentali per controllare
contemporaneamente la tensione di snervamento e la \eng{Input 
Transition Temperature}. Nello specifico:
\begin{equation}
\searrow d \Rightarrow \nearrow R_s \text{ e } \searrow I.T.T.
\label{eqn:RelMigli}
\end{equation}

\begin{figure}
\centering
\includegraphics[width = 0.5\textwidth]{MiglDimGrano}
\caption{Miglioramento tramite dimensione del grano}
\label{fig:MiglDimGrano}
\end{figure}

\begin{center}
\emph{Come si può controllare le dimensioni del grano ferritico?}
\end{center}
In genere non è sempre possibile agire sulla velocità di raffreddamento, 
perché nella laminazione a caldo, il raffreddamento, vine eseguito in 
aria. Perciò dipende dallo spessore del laminato.
Allora si può controllare la dimensione del grano austenitico durante 
la permanenza ad alta temperatura.
Perché il grano ferritico nuclea a bordo del grano austenitico, quindi 
se si limitano le dimensioni dell'austenite se ne limita la nucleazione
in ferrite, almeno diemensionalmente parlando.
Inoltre si può ricorrere all'aggiunta in lega di microleganti
che formano dei carburi i carbonitruri che "fissano" il grano
austenitico impedendone la crescita.

Da queste considerazioni si sono sviluppati gli \texttt{HSLA} ovvero
acciai ad alta resistenza ma basso legati.

\section{HSLA}
Gli \ac{HSLA} sono acciai con un prezzo più vicino a quello degli acciai 
al carbonio per via del fatto che non contengono tenori di elementi in 
lega eccessivamente alti. Inoltre la loro produzione non è eccessivamente 
costosa per via dei sistemi di rinforzo che sono stati accennati in 
precedenza.

Generalmente ne esistono diverse categorie perché sono venduti in base
alle loro caratteristiche meccaniche piuttosto che sulla composizione
chimica:
\begin{itemize}
\item Acciai a migliorata resistenza alla corrosione atmosferica
\item Acciai microlegati ferritico-perlitici
\item Acciai con ferrite aciculare
\item Acciai con morfologia controllata delle inclusioni
\end{itemize}

\newpage
\subsection{Ferritico-Perlitici ad alta resistenza}
Sono considerati tra i primi \ac{HSLA}.

Il loro sviluppo è stato possibile grazie all'introduzione della laminazione in controllo.

\begin{definition}{Laminazione in controllo}{*}
Deformazione a caldo con controllo accurato della temperature,
in modo da bilanciare l'effetto dell'affinamento del grano per
deformazione con quello della ricristallizzazione dovuta alle 
alte temperature.
\end{definition}

L'evoluzione di questi saranno gli acciai ferritico-perlitici
ad alta resistenza e tenacità.

Sono acciai che contengono V e Nb come elementi microalliganti.
Per questi acciai le temperature di fine laminazione sono in genere 
superiori ai $900\unit{\celsius}$ e il raffreddamento avviene in
aria calma.
Si ottiene, dunque, un acciaio con grano ferritico non in controllo ma
rinforzato dai carburi dei microlegati.
\begin{description}
\item[Acciai al niobio] interessa la precipitazione dei carburi NbC,
\item[Acciai al vanadio-azoto] interessa la precipitazione dei nitruri
VN.
\end{description}

\begin{wrapfloat}{figure}{O}{0pt}
\includegraphics[width = 0.5\textwidth]{FerPerHS}
\caption{Variazioni dello snervamento con l'aggiunta di microleganti}
\label{fig:FerPerHS}
\end{wrapfloat}

Dalla figura \ref{fig:FerPerHS} si evidenzia come per acciai microlegati
al Nb l'effetto del azoto sia indifferente. Già più marcato per gli 
acciai al V. Questo perché si formano i nitruri di vanadio.
Da notare come a parità degli altri leganti, gli acciai che presentano
microleganti ottengano una tensione di snervamento decisamente migliore.
Si vuole ricordare che non sono state eseguite operazioni di affinamento 
del grano, si considera infatti una dimensione del grano pari a $d 
\approx 14 \div 16\unit{\um}$.

Ulteriore miglioramento delle caratteristiche di snervamento si 
raggiunge nel caso di aggiunta di $1\%Cu + 1\%Ni$.
Il rame è poco solubile col ferro al diminuire della temperatura.
Se il materiale viene raffreddato molto rapidamente, il rame tende a 
rimanere in soluzione solida sovrassatura.
Operando un rinvenimento a $\approx 550\unit{\celsius}$ si ha la 
precipitazione della fase $\epsilon$ con ulteriore rafforzamento per 
precipitazione.
Rafforzamento inizialmente possibile solo per lamiere.


\subsection{Ferritico-Perlitici ad alta resistenza e alta tenacità}
Sono comunque acciai microlegati con cui si procede con 
\textbf{Affinamento del grano}, controllato tramite controllo della
dimensione del grano austenitico di partenza.
Grazie a questo processo si ha un miglioramento della tenacità.
Dopodiché si hanno due possibili vie per migliorare anche la resistenza:
\begin{enumerate}
\item \textbf{Normalizzazione}, in genere più adatta per lamiere con 
spessore $>25\unit{\mm}$;
\item \textbf{Laminazione controllata} per lamiere di spessore 
$<25\unit{\mm}$.
\end{enumerate}

\subsubsection*{Normalizzazione} per lamiere di spessore $>25\unit{\mm}$
In questo caso si segue un procedimento del tipo proposto alla figura \ref{fig:NormFerPer}.
I dettagli delle operazioni post laminazione a caldo saranno:

\setlength{\columnsep}{35pt}
\begin{multicols}{2}
\begin{definition}{}{*}
T finale di laminazione $> 900 \div 1000\unit{\celsius}$
\end{definition}
L'austenite non diventa sovrassatura degli elementi microlegati.
Dunque non si ha precipitazione di carburi per difficoltà nella
nucleazione. Abbassando la temperatura l'austenite cristallizza quando
il grano non è in controllo. Si andrà a generare un grano ferritico non 
particolarmente fine però rinforzato dai precipitati.
In generale si avrà \textbf{una resistenza non particolarmente alta e
tenacità scadente}.
\columnbreak
\begin{definition}{}{*}
T finale di laminazione $= 880 \div 900\unit{\celsius}$
\end{definition}
I carburi precipitano mentre l'austenite tende a riscristallizzare, dunque
ne controllano la crescita.
I precipitati devono essere molto fini e dispersi, altrimenti 
infragiliscono la struttura.
Si genererà un grano ferritico con $d \approx 5 \div 6\unit{\um}$ con 
contributo di $R_s = 390\div450\unit{\MPa}$.
Aggiungendo il contributo dato dai precipitati si arriva a $R_s = 
550\unit{\MPa}$. Ottenendo \textbf{resistenza e tenacità più alte}
\end{multicols}

\begin{figure}
\centering
\subfloat[][\emph{Possibilità sul trattamento di normalizzaizone}\label{fig:NormFerPer}]
{\includegraphics[width = \textwidth]{NormFerPer}}\\
\subfloat[][\emph{Possibili processi di laminazione controllata}\label{fig:LamiFerPer}]
{\includegraphics[width = \textwidth]{LamiFerPer}}
\caption{Lavorazioni per acciai ferritico-perlitici ad alta resistenza e tenacità}
\label{fig:LavHSHT}
\end{figure}

\subsubsection*{Laminazione controllata}
Anche in questo caso si possono realizzare due tipi di lavorazione come si vede
dalla figura \ref{fig:LamiFerPer}. Nello specifico: si evidenziano due diverse
tipologie di acciaio:
\begin{description}
\item[Primo tipo] Come già illustrato dalla figura \ref{fig:LamiFerPer}, i 
carburi presenti nella soluzione vengono mandati completamente in soluzione.
Per cui il materiale viene lavorato in laminazione ad una temperatura di 
$\approx 1250\unit{\celsius}$. Allora il processo, illustrato alla figura \ref{fig:1TipoferPer} diventa:
	\begin{description}
	\item[I Laminazione] Dal materiale caldo si susseguono una serie di passate
	tanto da portare il laminando ad una dimensione 5 volte più piccola di quella
	di partenza. Si ha la ricristallizzazione dell'austenite.
	\item[Attesa] Si lascia raffreddare il materiale fino a $\approx 900\unit{\celsius}$
	\item[II Laminazione] Si effettua una seconda laminazione ad una temperatura
	controllata tra $900\unit{\celsius} \div 750\unit{\celsius}$ dove si riduce lo spessore
	fino all' 80\%. Si ha precipitazione dei carburi mentre il grano austenitico si allunga
	senza ricristallizzare.
	\item[Fine lavorazione] Continuando il raffreddamento (in aria calma), si ha 
	un'alta velocità di nucleazione di ferrite ad alto rapporto superficie/volume.
	\end{description}
Ulteriori considerazioni per questo tipo di acciai. 
Confrontando con il caso della realizzazione di acciai ferritico-perlitici 
ad alta resistenza e tenacità tramite normalizzazione: in quel caso viene
limitata la crescita del grano austenitico equiassico.
Nel caso della laminazione controllata: si ricerca un grano austenitico a forma
più favorevole per la nucleazione ad alta velocità della ferrite.
\item[Secondo tipo] per questa tipologia, i carburi non vengono mandati completamente
in soluzione. Dunque:
	\begin{itemize}
	\item I carbonitruri non si disciolgono completamente alla temperatura di austenitizzazione.
	\item Siccome si sta lavorando a temperatura più basse, si avrà un'austenite satura a
	temperatura più bassa: dunque si ha un minor grado di sovrassaturazione dei microleganti.
	\item La ricristalizzazione risulterà impedita solamente nelle ultime passate di laminazione
	\item Il grano austenitico risulta controllato in fora mista tra volume e forma.
	\end{itemize}
Dunque si ottiene un acciaio con diverso rapporto tenacità/resistenza.
Si può aumentare ulteriormente la resistenza meccanica tramite incrudimento in
lavorazione a temperatura controllata di circa $600\unit{\celsius}$ andando a formare
una tessitura simile a quella della laminazione a freddo con conseguente anisotropia 
(circa il $5 \div 10\%$).
\end{description}

\begin{figure}
\centering
\subfloat[][\emph{Laminazione controllata del I Tipo}\label{fig:1TipoferPer}]
{\includegraphics[width = 0.4\textwidth]{1TipoFerPer}}\quad
\subfloat[][\emph{Microstruttura al microscopio ottico di HSLA al V}\label{fig:2TipoFerPer}]
{\includegraphics[width = 0.4\textwidth]{2TipoFerPer}}\\
\subfloat[][\emph{Ingrandimenti tramite micrografia TEM dei precipitati nanometrici di V}\label{fig:MicroStruttHSLA}]
{\includegraphics[width = \textwidth]{MicroStruttHSLA}}
\caption{Acciai Ferritico-Perlitici ad alta resistenza e tenacità}
\label{fig:FerPerHSHR}
\end{figure}

\newpage
\subsection{Acciai a basso tenore di C con struttura aciculare}
Con questa classe di acciai si è superato il valore di resistenza $R_s \approx 500 \div 600\unit{\MPa}$
senza ricorrere ad acciai da bonifica. Andando ad evolvere ulteriormente le classi
di acciai viste in precedenza.
In particolare questi acciai possiedono:
\begin{itemize}
\item Alta tenacità
\item Buona saldabilità
\end{itemize}
Caratteristiche donate dal fatto di una laminazione a caldo senza trattamenti termici che dona
alta tenacità e resistenza. Il tenore C è limitato, per cui hanno una buona saldabilità.
La struttura è quasi completamente \textit{bainitica}  (detta anche ferrite aciculare). 
Gli sviluppi si possono osservare dalle curve CCC, mostrate in figura \ref{fig:ConfFer-Per/LowC}.
Si osserva come l'aggiunta di Mo e B modifichino le curve. Andando a preferire una 
specifica struttura.
Infatti quei due elementi leganti Aumentano considerevolmente la temprabilità del acciaio ottenendo
proprietà meccaniche molto interessanti come: $R = 890\unit{\MPa}$, $R_s = 620\unit{\MPa}$, $A = 24\%$
con scarsa tenacità.
Ciò è dovuto alla presenza del B, che tende a segregare a bordo grano austenitico disattivando i centri 
di nucleazione di ferrite e perlite. Inducendo \textbf{frattura intergranulare}.

Si ha la necessità di sostituire il B con altri elementi meno problematici.

\begin{figure}
\centering
\includegraphics[width = \textwidth]{LowCAcic}
\caption{Confronto curve CCC tra acciaio Ferritico-Perlitico e a basso carbonio a struttura Aciculare}
\label{fig:ConfFer-Per/LowC}
\end{figure}

Successivi miglioramenti a questi acciai sono stati:
\begin{enumerate}
\item Sostituendo il B con il Nb (tenendo un tenore inferiore al $0.02\%$)
	\begin{itemize}
	\item Si hanno dei precipitati di NbC lungo i piani paralleli all'interno dei grani
	durante la trasformazione da austenite a ferrite;
	\item impedisce la crescita equiassica dei nuclei di ferrite. 
	\end{itemize}
\item Si possono ottenere delle resistenze più alte allo stato grezzo di laminazione a caldo 
sostituendo la Mn al Mo.
	\begin{itemize}
	\item Si favorisce la formazione di bainiti inferiori;
	\item $R_s \approx 620 \div 790\unit{\MPa}$;
	\item ci sono difficoltà operative ne produrre acciai con $Mn > 2\%$.
	\end{itemize}
\item Si può ottenere bainite anche con una tempra in linea all'uscita del laminatoio
mantenendo una velocità di raffreddamento $V_{raff} = 100\unit{\celsius/s}$
\item Migliorando le tecniche di produzione con limitazione spinta del tenore di zolfo e
modifica della forma delle inclusioni.
\end{enumerate}

Gli acciai \ac{HSS} sono acciai con $300\unit{\MPa} < R_m < 700\unit{\MPa}$ che includono gli
acciai rinforzati per incrudimento, soluzione solida, precipitazione e affinamento del grano
tra cui \ac{IF-HS}, \ac{IS}, \ac{BH}, \ac{CMn} e \ac{HSLA}.

Mentre gli \ac{AHSS} di prima generazione sono acciai con $R_m > 500\unit{\MPa}$ e costituiti in 
generale da una microstruttura complessa formata da più fasi (ferrite, martensite, bainite
e austenite residua) 
\begin{itemize}
\item Acciai rinforzati per trasformazioni strutturali (\ac{DP}, \ac{TRIP}, \ac{MS});
\item Acciai con rafforzamento misto: affinamento del grano, precipitazione e trasformazioni 
strutturali (\ac{CP}).
\end{itemize}

\begin{wrapfloat}{figure}{O}{0pt}
\includegraphics[width=0.8\textwidth, angle=90]{HSSAHSS}
\caption{Rappresentazione della resistenza/duttilità degli acciai HSS e AHSS}
\label{fig:HSSAHSS}
\end{wrapfloat}

La nomenclatura di questi acciai può essere rappresentata tramite la classica designazione 
\texttt{CEI} \texttt{EN}, però non è raro trovarli secondo la nomenclatura della \ac{ULSAB-AVC}.
La nomenclatura prevede:

\begin{definition}{Nomenclatura \ac{ULSAB-AVC}}{NomULSAB-AVC}
\begin{equation}
XXXX \: aaa/bbb
\end{equation}
Dove:
\begin{description}
\item[$XXXX$] indica il tipo di acciaio
\item[$aaa$] indica $R_s$ minimo garantito in $\unit{\MPa}$
\item[$bbb$] indica $R_m$ minimo garantito in $\unit{\MPa}$
\end{description}
\end{definition}

\begin{figure}
\centering
\subfloat[][\emph{Designazione acciai HSS \eng{cold rolled}}\label{fig:DesHSSCold}]
{\includegraphics[width = 0.8\textwidth]{DesHSSCold}}\\
\subfloat[][\emph{Designazione acciai HSS \eng{hot rolled}}\label{fig:DesHSSHot}]
{\includegraphics[width = 0.8\textwidth]{DesHSSHot}}\\
\subfloat[][\emph{Designazione acciai \eng{Dual Phase}}\label{fig:DesDP}]
{\includegraphics[width = 0.7\textwidth]{DesDP}}
\caption{Designazioni secondo \ac{ULSAB-AVC} a confronto con normative \texttt{UNI} \texttt{EN}}
\label{fig:ConfDeignazioni}
\end{figure}

Sebbene gli acciai visti in questo capitolo già riescano ad avere degli ottimi risultati in termini 
di resistenza meccanica, senza compromettere troppo la duttilità del materiale, ulteriori
sviluppi di questi acciai sono in fase di studio portando la così detta II generazione di 
acciai \ac{AHSS}.
L'obbiettivo è sempre quello di realizzare degli acciai a basso costo (basso legati) che 
possano essere realizzati in processi di fonderia non eccessivamente complicati.
In modo tale da permettere la realizzazione di acciai ad alta resistenza e tenacità a 
basso costo di produzione.

Ora verranno presentati alcuni tra gli \ac{AHSS} più avanzati presenti sul mercato come i
\ac{DP}, \ac{TRIP}, \ac{TWIP} e \ac{QandP}.

\begin{quote}
\emph{\textbf{Come si possono realizzare degli acciai che contengano più fasi 
nella soluzione solida?}}
\end{quote}

Il procedimento viene chiamato \textbf{processo termomeccanico} un esempio è riportato alla
figura \ref{fig:Termomeccanico}.
Si tratta di processi di produzione tramite cottura inter-critica e inter-critica continua.
Ovvero, il tutto parte sulla base delle curve \ac{TTT}. Per cui si realizza un raffreddamento 
controllato in modo da portare l'acciaio in in una specifica fase ferrosa.
Da lì, controllando il raffreddamento si possono ottenere delle soluzioni solide a 
diversa concentrazione di fasi diverse in base alla permanenza, in
una delle zone di trasformazione di fase. 

\begin{figure}
\centering
\includegraphics[width = 0.7\textwidth]{Termomeccanico}
\caption{Esempio di processo termomeccanico con i relativi risultati}
\label{fig:Termomeccanico}
\end{figure}

\subsection{Dual Phase}
Gli acciai \eng{Dual Phase} (\ac{DP}) sono caratterizzati da una lega abbastanza povera.

\begin{definition}{Dual Phase}{DP}
\\
\begin{tabularx}{\textwidth}{X|X|X|X}
\toprule
\textbf{C} & \textbf{Cr, Mo} & \textbf{V} & \textbf{Nb}\\
$0.06\div 0.15\%$ & $\leq 0.4\%$ & $\leq 0.6\%$ & $\leq 0.04\%$\\
\bottomrule
\end{tabularx}
\end{definition}
Le caratteristiche salienti per questo tipo di acciaio sono ad esempio:
la presenza di snervamento continuo, con un basso rapporto $R_s/R_m = 0.60\div 0.65$ 
(considerando che in genere per un comune \ac{HSLA} è $= 0.7\div 0.8$).
I prodotti ottenibili presentano una variegata possibilità di tensioni di snervamento
con $R_s = 500 \div 1200\unit{\MPa}$ anche se quelli più utilizzati sono i \texttt{DP600}.
Hanno un buon comportamento a deformazione che si presenta molto uniforme. Anche la resistenza a
fatica è molto valida. Come anticipato: la lega è molto povera quindi il costo di produzione non
è eccessivo.
In genere vengono impiegati per: parti resistenti delle carrozzerie e costruzioni metalliche 
generiche.

\begin{figure}
\centering
\includegraphics[width = 0.3\textwidth]{DP}
\caption{Microstruttura acciai \ac{DP}}
\label{fig:DP}
\end{figure}   

In generale per gli acciai \texttt{DP600}, in figura \ref{fig:DP}, si realizzano con una percentuale di ferrite attorno al
$60\div90\%$ che risulta molto povera di carbonio $<0.02\%$. Mentre la martensite è più ricca di
carbonio rispetto all'acciaio di partenza. Se ne ottiene:
\begin{enumerate}
\item Una struttura profondamente stampabile, grazie alla ferrite;
\item Con elevate caratteristiche meccaniche per via della martensite.
\end{enumerate}

Per ottenere questo tipo di acciaio viene eseguita una \textbf{ricottura continua intercrtitica}:
\begin{enumerate}
\item Dopo la laminazione l'acciaio viene portato a temperatura tra $A_1$ e $A_3$.
\item viene mantenuto a tale temperatura (in zona intercritica $\alpha + \gamma$) per la \
trasformazione da austenite a ferrite fino al 80\%.
\item Approfittando di un campo di non trasformazione tra ferrite e perlite e la bainite, si
effettua l'avvolgimento in \eng{coils} senza incontrare la trasformazione bainitica.
\end{enumerate}
Siccome resta la percentuale di austenite $\approx 20\%$, quella verrà trasformata in:
\begin{description}
\item[Bainite] circa il $2\div 5\%$;
\item[Martensite-$\alpha'$] circa il $18 \div 15\%$.
\end{description} 

\subsection{Acciai TRIP}
Prevedono anche una quota parte di austenite residua (tra il $5\div20\%$ di residuo), viene 
sfruttata per l'effetto \ac{TRIP} in figura \ref{fig:TRIPEffect}.
Deformando plasticamente l'austenite che diventa martensite, viene promossa tramite la deformazione 
del materiale. Dopo la deformazione diventa più resistente.
Dunque si aumenta la formabilità del materiale. Se si lascia dell'austenite residua, può essere che 
un forte impatto vada a trasformare la restate austenite in martensite. Portando il materiale a 
resistere all'impatto.

\begin{definition}{Acciai TRIP}{TRIP}
Tipica lega per gli acciai \ac{TRIP}:
\begin{tabularx}{\textwidth}{XXX}
\toprule
C $= 0.2\%$, Mn $= 1.8\%$, Si $= 1.6\%$ &
+Si e Al per stabilizzare l'austenite &
+ Ti e V per aumentare la resistenza\\
\bottomrule
\end{tabularx}
\end{definition}

\begin{figure}
\centering
\subfloat[][\emph{Composizione acciai TRIP}\label{fig:TRIPComposition}]
{\includegraphics[width = 0.3\textwidth]{TRIPComposition}}\quad
\subfloat[][\emph{Effetto \ac{TRIP}}\label{fig:TRIPEffect}]
{\includegraphics[width = 0.3\textwidth]{TRIPEffect}}\quad
\subfloat[][\emph{Microstruttura acciaio TRIP al microscopio}\label{fig:TRIP}]
{\includegraphics[width = 0.3\textwidth]{TRIP}}\quad
\caption{Gli acciai \ac{TRIP}}
\label{fig:AccTRIP}
\end{figure}

L'ottenimento degli acciai \ac{TRIP} non è troppo diverso da quello dei \ac{DP} visti in 
precedenza, cambia la zona di permanenza durante il raffreddamento come si vede in figura
\ref{fig:TRIPProduction}.

\begin{figure}
\centering
\includegraphics[width = \textwidth]{TRIPProduction}
\caption{Processo produttivo degli acciai \ac{TRIP}}
\label{fig:TRIPProduction}
\end{figure}

Sebbene le caratteristiche meccaniche siano eccezionali, questi acciai presentano dei limiti tipo:
\begin{itemize}
\item Hanno elevati valori di elementi in lega;
\item Le operazioni di saldatura sono rese difficili dagli alti tenori di Mn-Si o Mn-Al che 
chiedono altrettanto elevati tenori di C ($\approx 0.15\div 0.25\%$). Quindi rendendo le operazioni 
di saldatura abbastanza problematiche.
\item A causa degli elementi in lega, possono incorrere delle situazioni di ossidazione durante
le laminazioni e la zincatura a caldo non è incoraggiata.
\item Se la stabilità dell'austenite è modesta, possono verificarsi delle trasformazioni in 
martensite per via della bassa temperatura ambientale. 
\end{itemize}

\subsection{Acciai TWIP}
A differenza dei loro predecessori, presentano un comportamento allo snervamento particolare:
Mentre gli acciai \ac{TRIP}, se sottoposti a sollecitazione, trasformano i residui di 
austenite in martensite irrigidendosi. Gli acciai \ac{TWIP} resistono alla tensione fino a quasi
un 90\% di deformazione.
Ciò li rende particolarmente interessanti in quanto posseggono delle caratteristiche meccaniche 
molto promettenti, senza trascurare la tenacità.
Il problema di questi acciai deriva dal fatto che posseggono una lega particolarmente ricca.
Dunque non sono esattamente a buon mercato.

\begin{figure}
\centering
\includegraphics[width = 0.5\textwidth]{TWIP}
\caption{Confronto in termini di deformazione-snervamento tra acciai \ac{TRIP} e \ac{TWIP}}.
\label{fig:TWIP}
\end{figure}

\subsection{Q\&P}
Sono acciai realizzati tramite un innovativo metodo di trattamento termomeccanico.
Di base sono acciai \ac{TRIP} in cui viene controllata la diffusione del carbonio
dalla martensite all'austenite residua.

Trasferimento di carbonio dalla martensite all'austenite.
Il carbonio in eccesso non deve formare dei carburi,  ma deve trasferirsi nell'austenite. Perché 
quando si torna a raffreddare, l'austenite viene stabilizzata abbassandone la temperatura $M_f$.
In questo caso invece di avere un'austenite che si trasformerà in martensite, deve rimanere tale e 
quale per garantire più deformabilità, mantenendo una buona durezza.

\begin{figure}
\centering
\includegraphics[width = 0.5\textwidth]{QandP}
\caption{Metodo di realizzazione dei \ac{QandP}}
\label{fig:QandP}
\end{figure}

\begin{figure}
\centering
\includegraphics[width = \textwidth]{AHSSIIGen}
\caption{Acciai \ac{AHSS} di seconda generazione e la loro collocazione in caratteristiche 
meccaniche}
\label{fig:AHSSIIGen}
\end{figure}