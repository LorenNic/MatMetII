%%%%%%%%%%%%%%%%%%%%%%%%%%%%%%%%%%%%%%%%%%%%%%%%%%%%%%%%%%%%%%%%%%
% Capitoli già visti a lezione								    %
%%%%%%%%%%%%%%%%%%%%%%%%%%%%%%%%%%%%%%%%%%%%%%%%%%%%%%%%%%%%%%%%%%
\newpage
\section{Acciai inossidabili austenitici}
\subsection{Tattamenti termici applicabili}
Durante un'eventuale saldatura, il processo di raffreddamento non è 
controllato per cui non si è sicuri se il processo di stabilizzazione sia 
ancora valido. Perciò è meglio optare per degli acciai low carbon.
Per gli acciai austenitici la sensibilizzazione avviene ad una certa distanza 
dal cordone di saldatura. Per quelli feritici invece si avvicina al cordone
per via delle caratteristiche dell'acciaio.

\subsubsection{Trattamento di distensione}
Si riscalda l'acciaio ad una temperatura inferiore alla temperatura di inizio
sensibilizzazione. Tenendo il processo per circa $30min \div 2h$.
Successivo raffreddamento in aria calma. 
Si fa per eliminare le tensioni interne generate per qualsiasi motivo.
Ciò riduce il pericolo della \emph{tensocorrosione}.
Il processo si esegue solo in caso in cui si presume che ci sia tale moti vo 
di corrosione.

Gli acciai inox austenitici non hanno particolari caratteristiche meccaniche 
Risultano però molto deformabili e hanno una buona capacità di incrudimento.
È possibile che si vada a generare l'effetto \ac{TRIP} (visto in precedenza).
Per via del fatto che essendo deformato a freddo, c'è la tendenza che 
l'austenite venga trasformata in martensite (Sotto la temperatura $M_d$ 
ovvero quella temperatura che permette la trasformazione di austenite in
martensite a fronte di una deformazione).

\missingfigure{Inserire i grafici delle proprietà meccaniche degli 
austenitici}

Gli acciai austenitici hanno una buona resistenza al Creep ovvero
uno sforzo ad alta temperatura a carico costante.

Riassumendo:\\
Gli acciai inossidabili austenitici hanno un buon comportamento a bassa 
temperatura in quanto non presentano la transizione duttile-
fragile. Ad alta temperatura bisogna stare attenti alle temperature  
critiche di sesibilizzazione.
Possono resistere bene alla corrosione in ambienti aggressivi.
Sono molto utilizzati in quei campi dove la sicurezza è di 
fondamentale importanza.

I super-austenitici vengono chiamati così perché hanno valori di $PREN>40$.


\section{Acciai inossidabili ferritici}
Sono ancora acciai monofasici. Secondo l'\ac{AISI} sono 
designati come 4XX. Non è completamente indicativa della struttura in
quanto ci sono sia i ferritici che i martensitici con la stessa nomenclatura.

La tipica lega dei ferritici prevede leghe: Fe-C-Cr.
Dunque si scelgono questi acciai quando si vuole optare per una scelta più 
economica.
Hanno struttura ferritica a tutte le temperature. Fanno eccezione gli acciai
detti \emph{semiferritici} che se scaldati e portati in austenitizzazione 
e raffreddati velocemente formano della martensite.
In genere non sono adatti ai trattamenti termici. E sottoporli a tali 
trattamenti compromette la resistenza a corrosione.
Sono suscettibili al rinforzo per incrudimento, comunque meno di quanto fanno 
gli austenitici.
Un acciaio ferritico di riferimento è l'\texttt{\ac{AISI} 430}.
\missingfigure{Aggiungere Metallografia AISI 442}

\todo[inline]{Vedi la tabella della comparazione delle proprietà}

Possibile formazione della fase $\sigma$ che contiene molto cromo 
andando ad infragilire la struttura e sensibilizza alla corrosione 
intergranulare.

\subsection{Trattamenti termici}
Non si vanno a migliorare le caratteristiche meccaniche come per gli 
asutenitici. Piuttosto vengono tamponate delle situazioni in cui gli 
acciai potrebbero essere più sensibili.

\subsubsection{Ricottura di ricristallizzazione}
Si effettua per ricristallizzare un materiale che precedentemente ha subito 
una deformazione plastica.
Non si deve mai superare la temperatura critica dei $850\unit{\celsius}$ per 
cui il grano ferritico tenderebbe a ingrossare troppo.
Siccome stiamo effettuando una ricristallizzazione non è detto che si possa 
effettuare una successiva deformazione (soprattutto se il pezzo è già 
stato formato).
Si preferisce abbondare col raffreddamento, di solito fatto in acqua.
Perché si può incorrere in fenomeni di infragilimento.
\begin{description}
\item[Infragilimento a $457\unit{\celsius}$] Si ha una decomposizione della 
fase $\alpha$. Da lì inizia a formarsi, per via dell'alta percentuale di 
cromo, da cui si forma sia fase $\alpha$ che $\alpha'$.
Dove la prima è molto ricca di Fe, l'altra più ricca di Cr.
Se ne moficano le caratteristiche meccaniche e aumenta la TTDF%
\todo{acronimo}.
È un problema reversibile, si può scaldare nuovamente il materiale per 
poi raffreddarlo più velocemente.
\item[Infragilimento per fase $\sigma$] È reversibile, in maniera simile 
a quanto visto prima. Bisogna stare attenti alla temperatura di 
riscaldamento che deve essere $\approx 800\unit{\celsius}$. 
\end{description}

Ciò pone questi acciai in situazioni di sensibilizzazione
che può verificarsi a $T > 950\unit{\celsius}$. Perciò bisogna porre 
particolare attenzione alle saldature.

La motivazione è ancora in fase di studio. Perciò non è completamente chiara.
\missingfigure{Aggiungere i grafici della sensibilizzazione}
Si potizza che: a causa della non omogeneità della matrice del materiale, si 
vadano a formare delle isole austenitiche interne al grano ferritico.
Allora possono formarsi dei carburi di ferro che precipitano, risultando più 
sensibili alla corrosione.
Non esiste un processo di stabilizzazione del materiale, perché il processo
di sensibilizzazione non è strettamente legato alla precipitazione di 
carburi di cromo. Perciò non è così efficace.

\subsubsection{Acciai ferritici ELI}
Si tratta di leghe Fe-Cr-Mo a bassisimi tenori di C e N.
Presentano alta resistenza alla tensocorrosione.

\section{Acciai inossidabili Martensitici}
Si applicano per quelle applicazioni in cui si vuole una buona resistenza
alla corrosione (ben maggiore dei CORTEN) ma non si vuole rinunciare
alle caratteristiche meccaniche.
\todo{Aggiungere i tenori di carbonio}
Sono scuscettibili ai trattamenti termici di indurimento in quanto presentano 
i punti critici $A_1$ e $A_3$. 
Dopo la tempra presentano una struttura martensitica o martensitica con 
carburi. Per cui si raggiungono durezze molto elevate $\approx 45\unit{HRC}
\div 65\unit{HRC}$.

Non sono facili da saldare in quanto i tenori di carbonio sono parecchio 
elevati.

Una designazione tipica di riferimento è l'\texttt{AISI 410}.

\subsection{Trattamenti termici}
\todo[inline]{vedi slides per magiore completezza}

Hanno elevata temprabilità e vengono in genere temprati in olio e in aria, 
detti anche autotempranti (la scelta dipende dagli spessori).

\section{Acciai inossidabili PH}

Acciai non più monofasici, andando ad interrogare gli acciai duplex.
\emph{Precipitation Hardening}. 
Sono stati sviluppati per richieste dell'industria aeronautica e bellica.
Hanno elevate caratteristiche meccaniche al contempo elevate caratteristiche
alla corrosione.
Tali caratteristiche si sono ottenute grazie al processo di indurimento per
precipitazione.
Spesso vengono nominati tramite il nome commerciale piuttosto della
designazione normativa.
Secondo l'AISI \todo{Acronimo} sono la serie 6XX. Come riferimento al
\texttt{AISI 630}.

Si nota, dalla nomenclatura europea, che i tenori di carbonio è molto
molto basso $0.01\%$ per alcune formulazioni.
Questa caratteristica non è troppo distante dagli acciai austenitici.
Ciò impone che le caratteristiche meccaniche vengano demandate ai (molti)
elementi in lega. Si demanda alla precipitazione di altre soluzioni.
Ad esempio spesso si hanno precipitazioni tipo: $Ni_3X$ dove $X$ è un
altro elemento in lega.
