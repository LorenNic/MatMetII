\chapter{Acciai da utensile}\label{chp:Utensili}
Sono utilizzati per la lavorazione o la generazioen della forma dei materiali metallici e non.
Le lavorazioni a cui devono adempire possono essere:
\begin{itemize}
\item Asportazione di materiale
\item Deformazione plastica a caldo
\item Deformazione plastica a freddo
\end{itemize}

Le principali proprietà spesso contrastanti fra le quali trovare un compromesso sono;
\begin{itemize}
\item durezza,
\item resistenza all'usura e all'abrasione,
\item mancanza di fragilità
\end{itemize}

Spesso per gli acciai da utensili si stimano in base alla loro durata: tanto più riesce a mantenere le sue caratteristica, meglio è.
In molti casi non viene specificata un parametro definito. Si cerca di farli durare il più possibile donandogli determinate caratteristiche gli permettono di sopportare una maggiore usura a fronte della lavorazione per cui sono studiati.

\begin{quote}
Quanti tipi di acciai per utensili esistono?
\end{quote}

Ne esistono di molteplici formazioni, leghe ecc\dots in base alle necessità della lavorazione.
Alcuni sono più adatti per la lavorazione a freddo, alcuni mantengono una durezza residua ad alta temperatura (temperatura di flash) nel tempo, alta resistenza all'abrasione, ecc\dots
Caso per caso si hanno esigenze diverse per cui si sono sviluppati acciai corrispondenti.
Tipo:
\begin{description}
\item[Acciai a basso C] per esempio il C40U, è un acciaio a basso tenore di carbonio ottimizzato per la realizzazione da utensili.
\item[Acciai debolmente legati] Anche in questo caso ottimizzati in base alla lavorazione che devono eseguire
\item[Acciai fortemente legati] Si parla di un tenore di elementi leganti che arriva fino al 30\%.
\end{description}

Di particolare importanza diventano i trattamenti termici, in base a questi uno stesso acciaio può ottenere proprietà diverse.
Per esempio, gli acciai rapidi sono acciai che possono lavorare altri acciai, ottimizzati per lavorare ad alta velocità.

Di particolare importanza per questi acciai è sicuramente la purezza e la sequenza delle fasi di produzione dell'acciaio.

Il rendimento dell'utensile dipende particolarmente da:
\begin{enumerate}
\item Appropriata progettazione dell'utensile,
\item Accuratezza nella produzione dell'utensile,
\item Selezione accurata dell'acciaio più adatto allo scopo,
\item Corretta esecuzione del trattamento termico più appropriato.
\end{enumerate}

\todo[inline]{Citazione delle proprietà più richieste}

La durezza dell'utensile deve essere maggiore del materiale da lavorare. Non è sempre vero, o meglio va adattato alla lavorazione. In genere si sceglie una differenza di durezza tra i due acciai di circa un centinaio di Viskers.
La resistenza all'usura è un discorso complesso, indicativamente dipende dall'affinità tra i due materiali. Se i due metalli lo sono, l'utensile tenderà a durare meno.
La resistenza a fatica viene suddivisa tra:
\begin{description}
\item[Resistenza a fatica meccanica] in cui per ovviare a tale problema si agisce più sulla conformazione dell'utensile piuttosto del materiale.
\item[Resistenza a fatica termica] Dovuto ai cicli di produzione, ad esempio in presso-fusione dell'alluminio, in cui ne risente la finitura superficiale. Vale anche nel caso di utensili per asportazione. Prevalentemente è più problematica questa fatica perché gli utensili vengono più compressi che non trazionati.
\end{description}
Anche per questi acciai valgono le classiche relazioni tra durezza-duttilità e tenacità.
la temprabilità non è da sottovalutare sempre in termini di garantire delel caratteristiche meccaniche non indifferenti per questi acciai.
Attenzione all'aggiunta "compulsiva" di elementi di lega per aumentare la temprabilità. Di solito si abbassa la conduttività termica: si rischia di andare a formare delle cricche a caldo sia in salita di temperatura che in spegnimento. Per cui è necessario controllare il mezzo di spegnimento.
Ingrossando il grano si elimina una delle poche strategie per tenere duttile l'acciaio. Gli elementi in lega tendono ad irrigidire molto l'acciaio perdendo di tenacità. Per cui ingrossare troppo i grani per surriscaldamento non è cosa buona.

Negli acciai da utensili gli elementi in lega possono essere divisi in 2 gruppi:
\begin{description}
\item[Elementi che formano carburi] Cr, Mo, W, V, Ti
\item[Elementi che non formano carburi] Si, Ni, Mn, Co
\end{description}
\todo{\\Riorganizzare La descrizione precedente per abbellirla}
I vari elementi non vengono mai usati da soli ma agiscono in maniera sinergica.
Molti degli elementi possono avere circa lo stesso effetto: la scelta tra i vari può essere fatto su prove sperimentali o on base al costo della lega.

I carburi che si formano sia sostituendo il Fe nella cementite che tramite carburi particolari tipo: $MC$, $M_2C$, $M_3C$, $M_6C$, $M_7C_3$\dots
Tutti questi carburi sono più duri della cementite:
\begin{itemize}
\item aumentano la durezza e la resistenza all'usura
\item rallentano l'addolcimento durante il rinvenimento, fondamentale per gli acciai per utensili per lavorazioni a caldo 
\item I carburi hanno una minore velocità di coalescenza inferiore rispetto alla cementite.
\end{itemize}
Per avere la massima precipitazione fine e dispersa di carburi è necessaria la massima dissoluzione del carbonio nell'austenite.

\missingtable{Tabella tipologie di carburi ottenibili}
\missingfigure{Diagramma resistenza carburi}
Ai carburi è richiesto di avere un periodo di coalescenza il più lungo possibile.

Gli acciai da utensili possono presentare il \texttt{Picco di durezza secondaria} ovvero un picco di durezza, alle volte superiore, per via delle alte capacità di resistenza al rinvenimento. Ciò è dovuto grazie alla precipitazione di alcuni carburi che avendo una velocità di diffusione modesta, che avviene solamente per temperature alte, allora si ha la precipitazione dei carburi proprio a quelle temperature in cui rinviene: per cui aumenta la durezza generale dell'acciaio. 

Per gli elementi che non formano carburi
In particolare il Co non forma mai carburi solidi. In più, rallenta la velocità di rinvenimento rallentando anche tutte quelle trasformazioni che avvengono durante il rinvenimento. Riduce però la temprabilità.
\todo{\\Aggiungere anche gli altri elementi}

\section{Trattamenti termici}
Vediamo il tipico trattamento termico per gli acciai da utensili potrebbe essere:
\missingfigure{Grafico trattamento termico acciai utensili}
Ipotizzando di utilizzare un acciaio particolarmente legato.
Si parte dalla colata di una lingotto o lingottino%
\footnote{Si utilizzano dei piccoli stampi di colata perché alto legaggio corrisponde ad una bassa condittività termica. Quindi per evitare che il materiale si fratturi durante il raffreddamento allora si preferisce realizzare dei lingotti molto piccoli}
e lo si raffredda al interno dello stampo. 
In fase di prima solidificazione si potrebbero ottenere strutture dendritica che presenta una maglia inter-dendritica caratterizzata dalla presenza di carburi. I carburi in prima solidificazione sono i residui della fase di prima solidificazione. Saranno, dunque, generati direttamente dal liquido: detti anche \textit{primari}. Per sciogliere i carburi bisogna riportare il materiale ad una temperatura prossima a quella di fusione del materiale. Però non si vuole arrivare a quel punto perché se si manda l'acciaio vicino alla fusione si rischia di \emph{bruciare l'acciaio}.
A quello stato conviene rifondere tutto.
La maglia non la si vuole. Si ha una zona del materiale che è particolarmente dura riducendo la tenacità.
Allora si cerca di rompere la struttura dendritica e la maglia di carburi.
Il processo di deformazione plastica a caldo viene detto \textbf{impastamento}. Inoltre si vuole evitare un incolonnamento dei carburi: altrimenti si avrebbe un materiale decisamente anisotropo.
Segue una fase di ricottura, a temperatura più bassa della temperatura di lavorazione a caldo. Segue un raffreddamento controllato lente.
Si va a fare la formatura dell'utensile.
In fine di esegue una sorta di distensione per rilassare il materiale.
Dopo può seguire il trattamento termico per la specifica caratteristica ricercata.

Il trattamento termico successivo prevede tre fasi:
\missingfigure{Trattamento termico di indurimento}
\begin{enumerate}
\item Austenitizzazione,
\item Tempra,
\item Rinvenimento doppio o addirittura triplo.
\end{enumerate}
Il trattamento di indurimento viene eseguito post distensione perché data la bassa conduttività termica, se si scalda il materiale mentre presenta delle tensioni residue per via delle lavorazioni a freddo si rischia di creare delle fratture del materiale.

\todo[inline]{Aggiungi lezione del 21/04/2023}