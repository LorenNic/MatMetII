\chapter{Ghise}\label{chp:Ghise}
Le ghise sono dei prodotti ferrosi, contenenti un alto tenore di carbonio.
Cioò conferisce proprietà completamente diverse dagli acciai, permettendo di raggiungere lo stato liquido a temperature più basse $\approx 1200\unit{\celsius}$. Sono particolarmente adatte alla lavorazione per colata.
Iteressano tutti i settori ingegneristici. 
Si considera una ghisa con tenore di carbonio $2.5\% \leq \%C \leq 4.5\%$.
è anche notevolmente superiore al massimo di solubilità dell'austenite alla temperatura eutettica.
Data l'elevata fragilità, non sono in genere lavorabili per deformazione plastica né a caldo né a freddo.
Possiedono delle proprietà che possono essere variate in larga misura in relazione alla lega, in funzione del controllo della produzione ed eventuale trattamento termico.

\missingfigure{Altoforno}
Dall'altoforno si ottengono sempre delle ghise in prima colata a partire da:
\begin{itemize}
\item Minerale di ferro
\item \eng{Coke} metallurgico
\item Vari fondenti
\end{itemize}
Si hanno diverse trasformazioni nel tino del forno si ottengono;
\begin{description}
\item[Ghisa] che verrà successivamente:
	\begin{itemize}
	\item Produzione di acciaio, in ciclo integrato
	\item Realizzazione dei getti in ghisa
	\end{itemize}
\item[Scorie] che vengono utilizzate per la realizzazione di altri materiali.
\end{description}
Dopo il processo in altoforno dove si ottiene la prima ghisa, si passano ad altri processi che permettono di ottenere dei prodotti più elaborati.
Quasi la totalità di getti viene ottenuta tramite prodotta tramite ghise di seconda fusione.

\missingfigure{Grafico fonderie di seconda fusione}

\section{Il materiale}
\missingfigure{Inserire grafico Fe-C}

Il carbonio è presente siamo come cementite sia come grafite, il che porta ad avere differenze microstrutturali e di proprietà.

\begin{description}
\item[Cementite]\todo{\\Aggiungere}
\item[Grafite]
\end{description}
%Da quotare
Come si possono classificare le ghise?

\begin{enumerate}
\item In base allo stato in cui è presente il carbonio
	\begin{itemize}
	\item Ghise bianche: solo $Fe_3C$
	\item Ghise grafitiche: presenza anche di carbonio grafitico 
	\end{itemize}
\item Se grafitiche in base alla forma della grafite: lamellare, nodulare, sferoidale, vermiculare.
\item In base alle proprietà: es. \texttt{EN GJS 500-7} dove viene indicato il tipo di ghisa, il carico di rottura ed eventualmente l'allungamento percentuale a rottura.
\end{enumerate}

\begin{description}
\item[Ghisa bianca] Il carbonio si presenta in forma $Fe_3C$; durante la solidificazione e il raffreddamento segue il diagramma di stato metastabile Fe-C.
\item[Ghisa grigia lamellare] Sarà presente del carbonio libero sotto forma di lamelle (grafite lamellare): la maggior parte del carbonio (tutto in senso ingegneristico) precipita durante la solidificazione sotto forma di lamelle di grafite.
\item[Ghisa grigia sferoidale] Il carbonio è largamente non combinato, si presenta sotto forma di sferoidi regolari di grafiti. Gli sferoidi csi ottengono per opportuno trattamento del fuso grazie ad aggiunte di elementi come $Ce$\todo{\\Completare}
\item[Ghise malleabili] sono quasi in disuso, sostituite dalle sferoidali. Il carbonio si presenta quasi tutto sotto forma di particelle di geometria tondeggiante ed irregolare. Questa morfologia è ottenuta mediante \ac{TT} delle ghise bianche con lo scopo di conferire duttilità.
\item[Ghise conchigliate o temprate] sono delle ghise grigie, per il fatto che non vengono colate in stampo in terra ma in uno stampo metallico, la velocità di raffreddamento è molto più elevata per via del più alto scambio termico con lo stampo. Si favorisce il raffreddamento secondo il grafico metastabile. La pelle del colato sarà prevalenemente a fase cementite. Il cuore del materiale si hanno gradienti di raffreddamento più tranquilli, infatti si ottiene della comune ghisa grigia.
\item[Ghise legate] Si utilizzano dei leganti per modificare la struttura durante il raffreddamento. Si va a stabilizzare una determinata fase a temperatura ambiente.
\end{description}

\missingfigure{Micrografie ghisa grigia, sferoidale}

\section{Fattori di influenza}
I fattori che influenzano il modo di presentarsi del carbonio nelle ghise:
\begin{itemize}
\item Composizione chimica;
\item Eventuale trattamento della ghisa fusa, detto inoculazione;
\item Velocità di raffreddamento;
\item Eventuale \ac{TT}.
\end{itemize}

Elementi come il $Si$ e basse velocità di raffreddamento sono fattori grafitizzanti.

\missingfigure{Diagramma di stato Fe-C-Si}

Aggiungere, per esempio il 2\% di $Si$ permette di traslare tutto il grafico Fe-C verso sinistra. Di conseguenza sposta la percentuale di carbonio che da la composizione eutettoidica e alza l'eutettoide.
Questo permette anche di aumentare il divario tra struttura eutettica stabile e metastabile. Per cui durante il raffreddamento il materiale ha la tendenza a formare più grafite.\todo{\\Aggiungere punti importanti del grafico Fe-C-Si}
A questo proposito si è sviluppato un parametro di carbonio equivalente che tiene in considerazione di questo effetto del silicio che viene usato per considerare una ghisa ipoeutettica, ipereutettica ecc\dots

Ne corso di un generico raffreddamento si potrebbe raggiungere l'eutettico metastabile solo con un sottoraffreddamento molto più rapido di quello stabile. Quindi si ha un effetto grafitizzante.
Al contrario il $Cr$ ha un effetto \textbf{antigrafitizzante}.
\todo{\\Aggiungere gli elementi grafitizzanti e antigrafitizzanti}

Se si mantiene costante uguale a 1 il rapporto tra silicio e cromo si ha un intervallo di eutettico stabile e metastabile costante decrescente.
Il fosforo è vero che è un elemento grafitizzante ma permette di abbassare di molto la temperatura di solidificazione: aumentando la colabilità della lega.

\subsection{Il carbonio equivalente}
Senza silicio la composizione dell'eutettico è pari al 4.3\% di C. 
All'aumentare del tenore di silicio diminuisce il tenore di carbonio dell'eutettico
\begin{equation}
CE = \%C + \frac{\%Si}{3}
\label{eqn:CarbEquivGhise}
\end{equation}
Per esempio per ogni 1\% di Si, la composizione eutettica viene ridotta di 0.3\% di C come si può vedere dalla \eqref{eqn:CarbEquivGhise}.
Quindi quando $CE = 4.3\%$ allora la lega è eutettica. Ghise con uguale CE possono essere ottenute con diversi rapporti di $C$ e $Si$.
Ne caso di composizione ad alto tenore di fosforo, essendo anche lui un elemento grafitico, si aggiunge nella considerazione del CE come nell'equazione \eqref{eqn:CarbEquivGhiseFosforo}.
\begin{equation}
CE = \%C + \frac{\%Si}{3} + \frac{\%P}{3}
\label{eqn:CarbEquivGhiseFosforo}
\end{equation}
\missingfigure{Grafico posizionamento delle ghise in funzione del C-Si}

\subsection{Solidificazione e raffreddamento}
Considerando una ghisa ipoeutettoidica, $CE \lesssim 4.0\%$valutiamo il raffreddamento presentato al grafico 
\missingfigure{Grafico raffreddamento}

\todo[inline]{aggiungi slide considerazioni sul raffreddamento}

\subsection{Inoculazione}
Ha il compito di controllare il numero di celle eutettiche e le modalità di solidificazione (stabile-metastabile) privilegiando la fase grafitica.
Gli inoculati vengono utilizzati in presenza di elevati sottoraffreddamenti tali da provocare la formazione di cementite o alterare la distribuzione della grafite.
Gli inoculanti sono delle ferroleghe $Fe-Si + Ca + Ba + Al + Sr + Zr$
\todo{\\Porre come definizione o esempio per evidenziare.}
Questi formano dei composti che servono da nuclei eterogenei per la formazione della grafite.
Vengono versati in siviera di colata, in percentuali dello $0.3\div1\%$ (in più si ritarda l'inoculazione prima della colata, maggiore è l'effetto).
Bisogna considerare un tempo di "evanescenza" altrimenti le ferroleghe si disciolgono nella composizione del liquido di tutta la ghisa vanificando il tentativo di inoculare.
\missingfigure{Nucleazione grafite + Strutture grafite}
la grafite risulta molto suscettibile allo sfaldamento per via delle modalità di interconnessione sei vari strati di grafene: i vari strati sono legati tramite forze di Van Der Waals. Per cui i legami sono deboli ed è molto facile romperli.

Volendo osservare il cambio di fase specifico per ogni struttura che la ghisa a raggiungere, partiamo dall'austenite.

\section{Solidificazione e raffreddamento}
Si ha una prima formazine di dendriti di austenite primaria per via del sottoraffreddamento rispetto a $T_L$.

Ci sono molte teorie sulla nucleazione e sull'accrescimento dell'eutettico $\gamma - C$.
Una delle teorie più accreditate si basa sul concetto delle \textit{celle eutettiche}:
\begin{itemize}
\item Nuclei di grafite si formano nel liquido e nel liquido impoverito 
\todo{\\Completa}
\end{itemize}
\missingfigure{Modalità accrescimento grafite}
\todo[inline]{Recuperare il materiale}
Siccome il carbonio dovrebbe attraversare una fase $\gamma$ per passare dal liquido al nucleo di grafite.\todo{\\Fino a qua}

La grafite vermiculare si ottiene tramite tentativo di inoculazione sferoidale riuscito male o non completamente. Per cui il risultato è un vermicello invece della sfera attesa. Non è sempre una cosa negativa: alcune ghise vermiculari vengono tutt'ora usate per componentistica di pompe oleodinamiche a pressione. Viene anche chiamata ghisa compatta.

Dopo il completamento della solidificazione dell'eutettico, ci si ritrova con un materiale contenente austenite e grafite.
Raffreddando ulteriormente tra la temperatura eutettica e eutettoidica si ha espulsione del carbonio dall'austenite che andrà accumulato nelle vicinanze della grafite già solidificata. Si forma allora della ferrite nelle prossimità dei nuclei grafitici nel caso in cui la lega sia fortemente grafitizzante. Se invece la ghisa è in condizioni antigrafitizanti tende a nucleare cementite.
\missingfigure{Inizio nuclazione ferrite + micrografie}

Gli elementi in lega hanno una notevole influenza anche nel promuovere le diverse strutture al raffreddamento dopo solidificazione:
\begin{description}
\item[Promotori ferrite] Si, Al,
\item[Promotori perlite] Sn, Mn, Cr, Ni, Sb, Cu,
\item[Affinatori perlite] Ni, Mo, V,
\item[Effetti sinergici tra i vari elementi in lega]
\end{description}
Gli elementi in lega non vanno aggiunti singolarmente ma vanno soppesati tra loro per ottenere, giostrandosi tra la forcella degli elementi in lega permessi dalla normativa, specifiche caratteristiche meccaniche richieste dal cliente.

Tale spaziatura è influenzata dalle condizioni di raffreddamento del getto.
La distanza $\lambda$ tra le lamelle di perlite ha effetto sulla durezza.
\missingfigure{Grafico distanza lamelle - durezza}

\section{Classificazione delle ghise}
\subsection{Classificazione della grafite}
La grafite viene classificata di:
\begin{description}
\item[Forma] lamelle, sferoidi, compatta;
\item[Distribuzione] uniforme, interdendritica, a rosette, \dots
\item[Dimensioni] da $1\unit{\mm}$ a $0.015\unit{\mm}$
\end{description}
La classificazione viene normata da \texttt{ASTM A247 (1967)}, \texttt{UNI 3775 (1973)}, \texttt{UNI EN ISO 945-1:2018}.
La classificazione può essere fatta sia ad occhio con piccoli ingrandimenti, sia attraverso degli opportuni misuratori digitali che restituiscono i vari parametri secondo normativa.

La classificazione è in numeri romani che identificano le varie casistiche. Vanno da I a VI in base alla forma.
\missingfigure{Possibili forme grafite}
Per la distribuzione, nel solo caso I, si utilizzano delle lettere da A a E come mostrato in figura.
\missingfigure{Possibili distribuzioni grafite caso I}
Per quanto riguarda dimensione si usa una cifra da $1 \div 8$. Si valutano 
le dimensioni della lamella o dello sferoide.
Per gli sferoidi si misurano i diametri per quanto possibile, mentre per le lamelle si misura una specie di corda.
\todo{\\Aggiungi eventuali esempi di codifica della grafite}

\subsection{Classificazione della matrice}
Le possibili matrici possono essere:
\begin{description}
\item[Martensitica]\todo{\\come si ottiene}
\item[Austenite]\todo{\\come si ottiene}
\end{description}

\subsection{Classificazione complessiva}
la normativa di riferimento è la \texttt{UNI EN 1560:2011}
\todo[inline]{Da porre come definizione come fatto in precedenza}
\missingtable{Classificazione generale ghisa}
Qui di seguito portati alcuni degli esempi \todo{\\Aggiungi gli esempi}
Esiste anche una designazione numerica in forma 5.XXXX
\missingtable{Aggiungere nomenclatura numerica}

\section{Ghise bianche}
Hanno impiego non particolarmente interessante dal punto di vista ingegneristico per via del fatto che tutto il carbonio è recluso nella cementite, per cui queste ghise vengono usate soprattutto per applicazioni lavoranti in compressione.
Eventualmente vengono prodotte per ottenere le ghise malleabili. Anche se pure le ultime sono ormai soppiantate.
\missingfigure{Metallografie ghise bianche}

\section{Ghise grigie}
Sono tra i materiali ferrosi più utilizzati, soprattutto tra le ghise.
Il loro nome è dato dalla colorazione grigiastra assunta dalla loro superficie di frattura.
La loro struttura \todo{\\completa}

Sono identificate tramite EN + GJL + 100 dove le cifre rappresentano il valore di tensione a rottura del materiale. Non viene specificato l'eventuale allungamento elastico per via del fatto che raramente hanno un comportamento elastico.
\missingtable{Ghise grigie normate}
\missingfigure{metallografie ghise grigie}
Nella vecchia normativa si sfruttava una notazione più tecnica tipo \texttt{G25} o \texttt{G00}\footnote{Non era garantita alcuna proprietà meccanica}.
La sterlite è una struttura eutettoidica ternaria particolarmente carica di fosforo. ha una struttura simile alla ledeburite per via delle macchie a leopardo presenti nella struttura.

Proprietà meccaniche. Risulta pure difficile definire un modulo elastico per via delle bassissime deformazioni. Tant'è che il modulo elastico viene calcolato matematicamente e non attraverso snervamento.

\todo[inline]{Aggiungere trattamenti termici}

\section{Ghise Sferoidali}
\todo[inline]{Aggiungere le sferoidali}
Cerio non è un vero e proprio neutralizzante per neutralizzare elementi di lega che tenderebbero a formare grafite in forma lamellare invece di sferoidale.
Durante la solidificazione la grafite precipita sotto forma di sferoidi o noduli più o meno regolari.
Date le caratteristiche può sostituire acciai da costruzione con tutti i vantaggi portati dalle ghise.

\section{Produzione della ghisa}
Per la produzione delle ghise si effettua un processo simile alla deossidazione vista per la calmatura degli acciai, in questo caso si parla di desolforazione perché è un elemento favorente la formazione lamellare della grafite piuttosto della sferoidale.
\subsection{La sferoidizzazione}
\todo{Aggiungere il necessario}
COme si fa in fonderia:
\begin{itemize}
\item In siviera: siviera aperta, siviera coperta, \eng{wire feeder}
In siviera aperta bisogna prevedere che si perde tanto magnesion per via della sua alta reattività con l'ossigeno.
\item \eng{flow through}
\item \eng{In-hold}
\end{itemize}