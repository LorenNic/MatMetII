\chapter{Allumino e leghe per impieghi industriali}\label{chp:Alluminio}
Si parla di leghe leggere per via del fatto che a sostituzione del ferro si hanno Allumino, Magnesio, Titanio.

Si tratta dell'unico, tra i materiali non ferrosi, a insidiare il ruolo di leadership dell'acciaio tra i materiali per impieghi industriali.

In teoria l'alluminio è più presente sulla crosta terrestre del ferro, ma la sua scoperta ed impiego si ha solamento nel 1800.
Solo nel 1836 si è messo a punto il processo produttivo, tutt'oggi utilizzato, \eng{Hall-Heroult} per realizzare il processo per ottenere l'alluminio dall'ossido.
Ciò è dovuto al fatto che le temperature per ottenere l'alluminio sono decisamente molto alte. Per cui si è preferito una specie di elettrolisi per ottenere il materiale.
La sua diffusione ha uno sviluppo per via delle sue caratteristiche fisiche e meccaniche ma anche per le proprietà estetiche.

Il suo impiego è per il $40\%$ nel mondo della mobilità. Poi trova impieghi nel mondo costruttivo ed edilizio. Nel packaging, stampi, e normali materiali di consumo.

L'alluminio non può cambiare struttura alotropica. Per cui non è suscettibile a cambi di fase tramite trattamenti termici a differenza del ferro. Dunque si hanno delle potenzialità limitate da questo punto di vista.  
Ha una temperatura di fusione particolarmente basso per cui qualsiasi processo di fonderia può essere utilizzato con l'alluminio. In particolare per quelle tipologie di alluminio da fonderia. Ovviamente ciò implica che non si possono utilizzare a temperature particolarmente alte.
Le leghe di alluminio lavorano bene fino al massimo $200\unit{\celsius}$.
La densità risulta quasi un terzo di quella del ferro: a parità di volume si ha un terzo del peso.
La conduttività termica è decisamente molto elevata. Assieme alla conduttività elettrica, minore di quella del rame, ma pesando molto molto meno per via della minore densità, viene impiegato per la trasmissione di potenza elettrica sulle principali linee elettriche.
Parlando del modulo elastico, si passa da un modulo elastico del ferro di $210\unit{\MPa}$ ai soli $70\unit{\MPa}$ dell'alluminio. Si può migliorare tale situazione attraverso opportuni trattamenti della lega.
Anche il carico di rottura è particolarmente basso rispetto al ferro.
Il tutto viene bilanciato dalla densità molto più bassa.
\missingfigure{Tabella di confronto della trave}
Per quanto il modulo elastico rappresenta un problema non indifferente. Si può ovviare il problema grazie alla flessibilità tecnologica dell'alluminio.

ha un'eccellente duttilità per via dei cristalli realizzati a struttura cubica a facce centrata.
Ha una buona lavorabilità, anche se dipende da eventuali \ac{TT} successivi.
ha una buona resistenza alla corrosione atmosferica. Ciò dovuto al sottile strato di ossido che fa da protettivo. Eventualmente ci sono trattamenti superficiali per aumentare ulteriormente la resistenza.
Risulta saldabile grazie alle tecniche moderne di saldatura.

\paragraph*{La produzione}
L'alluminio primario, ovvero il materiale ottenuto dal minerale, ha necessitato della messa a punto della tecnica di estrazione.
Dalla Bauxite si ottiene l'allumina, tramite processo \eng{Bayer} ($Al_2O_3$) da cui si ottiene l'alluminio per via chimica col processo \eng{Hall-Herloult}. In fine si può legare l'alluminio per ottenere le leghe primarie.
Alternativamente, si possono realizzare delle leghe secondarie da riciclo accurato di lege a fine ciclo vita. Per dovere di cronaca si stima che il processo di riutilizzo delle leghe secondarie si usa solamente il $10\%$ dell'energia utilizzata per generare leghe di alluminio primarie.
Tutto dovuto al processo di ottenimento dell'alluminio che è particolarmente energivoro.
Oltretutto il rapporto di ottenimento del materiale sta da $4\unit{\kg}$ di Bauxite si ottengono $2\unit{\kg}$ di allumina da cui si ottengono $1\unit{\kg}$ di alluminio.

Il processo di \eng{Hall-Heroult} è illustrato alla figura 
\missingfigure{Processo produttivo}
Il processo di elettrolisi provoca la generazione di gas di scarico ($CO_2$ e $CO$) e precipitati di alluminio fuso.
Il tutto eseguito ad una temperatura di $950\unit{\celsius}$

Il problema del riutilizzo delle leghe secondarie è dovuto alle impurezze che vengono fuse assieme. Grazie alle ricerche è possibile che le leghe secondarie ottengono quasi gli stessi traguardi dell'alluminio primario.
In più il fabbisogno di alluminio è molto più alto di quello che si può ottenere col solo recupero dell'alluminio secondario.

\section{Caratteristiche meccaniche}
L'allumino puro presenta una scarsa resistenza meccanica e una bassa durezza.
Se necessito di particolari caratteristiche devo per forza legarlo per elevare alcune caratteristiche.
Risulta decisamente vantaggiosa nel caso si parli di resistenza meccanica specifica detta:
\begin{equation}
\frac{Stress}{Density}
\end{equation}
Se si vogliono aumentare le caratteristiche meccaniche bisogna legare l'alluminio.
Si parte dalle leghe madri da cui poi si ottengono altre caratteristiche per ulteriori trattamenti.
Aggiungendo alle leghe madri gli elementi correttivi, che agiscono su particolari aspetti caratteristici.

\section{Classificazione delle leghe di alluminio}
SI differenziano principalmente tra leghe da fonderia e leghe da deformazione plastica.
La differenza non è solamente in termini di normativa ma soprattutto per la quantità di elementi in lega.

\missingfigure{Schema suddivisione leghe}

Dalle leghe da fonderia si può suddividere in \eng{as casted} o da \ac{TT}.
Per le leghe da deformazione plastica si può incontrare la casistica dell'incrudimento per deformazione plastica, oppure la strada del \ac{TT}.
Resta il problema che l'incrudimento a freddo può portare anisotropia nel materiale. Tutti i processi di deformazione possono essere eseguiti a freddo, principalmente, ma anche a caldo.

In base ai principali leganti si possono definire i principali \textbf{sistemi di leghe}
\missingfigure{Sistemi di leghe}
Alcuni sistemi di leghe sono più adatti per \ac{TT} e incrudimento a freddo ; altri non sono così suscettibili a miglioramenti dovuti a \ac{TT}.
Si nota che il $Mg$ è un elemento molto importante per le leghe di alluminio. Tra l'altro vale anche il viceversa, si vedrà al capitolo dedicato\todo{\\Riferimento}.
Il magnesio vede un utilizzo al $30\%$ di leghe di magnesio, un'ulteriore $30\%$ per le leghe di alluminio e la restante per la realizzazione di ghise sferoidali, viste al capitolo \ref{chp:Ghise} a pagina \pageref{chp:Ghise}.

\subsection{Normative di riferimento}
Le principali normative suddividono le leghe di alluminio per le leghe adatte per le leghe da deformazione plastica (\texttt{UNI EN 573}) e per le leghe adatte alla fonderia (\texttt{UNI EN 1780}, \texttt{UNI EN 1706}).
\missingfigure{Suddivisione normative alluminio}

Anche per le leghe di allumino adatte alla defermazione plastica esiste una designazione numerica e una a simboli chimici.
\missingfigure{Designazioni}
Le normative \texttt{AA} prevedono solamente designazione nuemrica. Mentre le normative \texttt{UNI} prevedono entrambe le classificazioni.
In riferimento ad entrambe, c'è una corrispondenza 1:1 per entrambe perché \texttt{UNI} ha assorbito le designazioni di \texttt{AA}.

Per le leghe da fonderia cambia tutto.
\subsubsection{Normative europee}
\begin{definition}{Normative alluminio europee}{*}
Verranno indicate le varie normative:\\
\begin{tabularx}{\textwidth}{cXX}
\toprule
& Def. Plastica & Fonderia\\
\midrule
Designazione numerica & \texttt{UNI EN 573 - 1} sistema di designazione numerica & \texttt{UNI EN 1780 - 1} sistema di designazione numerica\\
\midrule
& \texttt{EN AW-XXXX} & \texttt{EN AY-XXXXX}\\
\midrule
& \begin{description}
\item[A] alluminio
\item[W] \eng{wrought}
\item[XXXX] Indice della specifica famiglia
\item[Lettere Agg.] Variazioni nazionali sulla lega 
\end{description}
&
\begin{description}
\item[A] alluminio
\item[Y] può essere B se è pane di alluminio, C se è getto di alluminio (ottenuto tramite fusione di leghe AB), M se è lega madre da cui partire per ottenere le altre famiglie di leghe. 
\item[XXXXX] Indice della specifica famiglia di lega.
\end{description}
\\
\bottomrule
\end{tabularx}
\\ Parte 2:\\
\begin{tabularx}{\textwidth}{cXX}
\toprule
& Def. Plastica & Fonderia\\
\midrule
Designazione chimica & \texttt{UNI EN 573 - 2} sistema di designazione su composizione chimica & \texttt{UNI EN 1780 - 2} sistema di designazione su composizione chimica\\
\midrule
&
viene consigliato l'utilizzo facoltativo di questa modalità di designazione
& Come per il caso delle deformazione plastica.\\
\bottomrule
\end{tabularx}
\\
Parte 3:\\
\begin{tabularx}{\textwidth}{cXX}
\toprule
& Def. Plastica & Fonderia\\
\midrule
Designazione chimica e utilizzo & \texttt{UNI EN 573 - 3} composizione chimica  e prodotti& \texttt{UNI EN 1706} composizione chimica e caratteristiche meccaniche dei getti\\
\bottomrule
\end{tabularx}
\end{definition}

\todo[inline]{Aggiungere la classificazione delle famiglie della UNI EN 573-1}

\missingtable{Designazione UNI EN 573-1}
\missingtable{Designazione UNI EN 573-2}
\missingtable{Designazione UNI EN 573-3}
\todo[inline]{Esempi di designazione}

In confronto con la normativa americana c'è un rapporto 1:1, per cui c'è una corrispondenza esatta tra i due metodi di designazione. Basta anteporre la signa \texttt{AA}.

\missingtable{Designazione UNI EN 1780-1}
\missingtable{Designazione UNI EN 1780-2}
Si nota che nella famiglia delle leghe da fonderia a base di silicio è molto variegata. Questo perché il silicio è un elemento fortemente colabilizzante: migliora la colabilità del fuso.

\missingtable{Designazione UNI EN 1706}
La norma specifica i limiti di composizione chimica delle leghe di alluminio per getti e le proprietà meccaniche delle provette colate a parte.
Si trova:
\begin{itemize}
\item Definizione dei principali processi di fonderia
\item Abbreviazioni utilizzate per designare i principali processi di fonderia
\item Designazione degli stati metallurgici delle leghe da fonderia
\item Indicazione sulla designazione che deve apparire a disegno.
\end{itemize}

\subsubsection{Normative americane}
Per le leghe da fonderia vengono utilizzati dei codici completamente differenti da quelle europee.
\missingtable{Designazione fonderia americana}

\texttt{EN AB-42100 [Al Si7Mg0.3]} $\rightarrow$ \texttt{A356.0}
La lettera davanti può essere
\begin{description}
\item[A] Tenore limite di ferro $\lesssim 0.15\%$
\item[B] Tenore limite di ferro $\lesssim A$
\item[C] Tenore limite di ferro $\lesssim B$
\end{description}

\subsection{Normative sulla base dei TT}
In tabella vengono riportati quale siano le famiglie trattabili termicamente oppure no.
\missingtable{Tabella sì o no TT}
In genere vale che se una lega non è trattabile termicamente allora è trattabile per incrudimento.
Tutte sono trattabili per soluzione solida.
In generale tutte le leghe sono trattabili per incrudimento. Certo è che se si vuole ottimizzare le proprietà con un \ac{TT} andare ad incrudire non è la strada migliore e il costo dei trattamenti rischia di superare i benefici dati dall'acquisti di quella specifica lega.

\todo[inline]{Aggiungi gli stati di incrudimento UNI EN 515}