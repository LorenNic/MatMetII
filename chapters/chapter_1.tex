\chapter{Classificazione e Designazione degli acciai}\label{chp:ClassAcc}
\section{La normazione}
Per cominciare, è utile osservare come gli enti di normazione descrivono gli acciai.
tra l'altro sono tra i prodotti più normati presenti sul mercato industriale.
Dapprima:
\begin{description}
\item[UNI] sigla che indica una normativa realizzata dall'Ente nazionale di Unificazione.
Ente che norma tutte le attività produttive sul mercato italiano. Inoltre è facente parte del \acs{CEN}. Difatti applica sul suolo italiano tutte le normative date dallo stesso \acs{CEN}.
Non è ammessa la presenza di normative che siano in contrasto con quelle europee.
\item[EN] contraddistingue le norma sviluppate dal \ac{CEN}.
Le normative EN devono essere percepite da tutti gli stati membri dello spazio economico europeo.
Ciò per garantire il libero scambio di prodotti al interno del mercato.
Il EN è composto dai principali enti nazionali di normazione degli stati membri nello spazio economico europeo.
\item[ISO] rappresenta tutte le normative sviluppate dal \ac{ISO}. Possono essere un riferimento applicabile per tutto il mondo. Una nazione può decidere se applicare le norma \acs{ISO} indipendentemente da quanto fatto dal \acs{CEN}.
\end{description} 

Secondo le normative della \acs{CEN} le normative hanno lo scopo di:
\begin{quote}
Stabilire le condizioni tecniche per lo scambio di prodotti e di servizi assicurando il continuo adeguamento allo sviluppo delle tecnologie e dei bisogni del mercato
\end{quote}
con lo scopo di eliminare le barriere commerciali, almeno tra gli stati europei.

Una prima classificazione dei tipi di acciai perché esistono tante classi di materiale.
Dunque si può pensare ad una divisione in base:
\begin{multicols}{2}
\begin{itemize}
\item composizione chimica;
\item processo di fabbricazione;
\item caratteristiche meccanico-fisiche e di impiego;
\columnbreak
\item costituenti strutturali;
\item ecc\dots
\end{itemize}
\end{multicols}

Non a caso sono stati citati i precedenti aspetti, in fatti le normative vanno a coprire gli aspetti stessi, come mostrato nella tabella \ref{tab:NormGen}

\begin{table}
\centering
\caption{Norme di carattere geneale}\label{tab:NormGen}
\begin{tabularx}{\textwidth}{>{\bfseries}lX}
\toprule
UNI EN 10020:2001 & Descrizione e classificazione dei tipi di acciaio\\
UNI EN 10027-1:2016 & Sistemi di designazione degli acciai, \texttt{Designazione alfanumerica}\\
UNI EN 10027-2:2015 & Sistemi di designazione degli acciai, \texttt{Designazione numerica}\\
UNI EN 10025-(1-6):2005 & Prodotti laminati a caldo di acciai per impieghi strutturali\\
UNI EN 10079:2007 & Descrizione dei prodotti di acciaio (forma, dimensioni, aspetto, stato superficiale)\\
\bottomrule
\end{tabularx}
\end{table}

Secondo la norma \texttt{UNI EN 10020:2001}:
\begin{quote}
L'acciaio è un materiale il cui \emph{tenore in massa di Ferro (Fe) è maggiore di quello di ciascuno degli altri elementi} ed il cui \emph{tenore di Carbonio (C) è generalmente minore del $2\%$} e che contiene altri elementi. Un numero limitato di acciai al Cromo (Cr) può avere tenore di carbonio maggiore del $2\%$, ma tale valore del $2\%$ è il tenore limite corrente che separa l'acciaio dalla ghisa.
\end{quote}
Sempre la stessa norma definisce la classificazione principale degli acciai \ref{fig:UNIEN10020:2001}.

\begin{figure}
\usetikzlibrary{trees}
\begin{tikzpicture}[
sibling distance = 10em,
every node/.style={rectangle, rounded corners, draw, align=center,}
]
\node{Acciai}
	child{ node[top color = UnifeLight] {Non legati}
		child{ node[bottom color = UnifeDark!80, white] {di qualità}}
		child{ node[bottom color = UnifeDark!80, white] {speciale}}}
	child{node[top color = UnifeLight] {Inossidabili}}
	child{node[top color = UnifeLight] {Legati}
		child{node[bottom color = UnifeDark!80, white] {di qialità}}
		child{node[bottom color = UnifeDark!80, white] {speciali}}};
\end{tikzpicture}
\caption{Suddivisione acciai in base alla normativa UNI EN 10020:2001}
\label{fig:UNIEN10020:2001}
\end{figure}
Dove:
\begin{itemize}
\item \textcolor{UnifeLight}{$\blacksquare$} è la suddivisione per composizione chimica;
\item \textcolor{UnifeDark!80}{$\blacksquare$} è la suddivisione in base alle caratteristiche meccanico-fisiche della suddivisione chimica.
\end{itemize}
L'appartenenza ad una classe si basa sulla composizione chimica di colata indicata sulla norma di prodotto, prendendo in considerazione il valore minimo.
Vediamo ora come vengono suddivise le categorie in base alla norma.
\begin{description}
\item[Acciai non legati] sono gli acciai per cui \emph{Nessuno dei valori limite, rigorosamente fissati dalla norma (tabella \ref{tab:Prosp1}), è raggiunto dai rispettivi tenori degli elementi in lega} (escluso il C).
\item[Acciai inossidabili] sono acciai contenenti \emph{almeno il $10.5\%$ di Cr e al massimo l'$1.2\%$ di C}.
\item[Acciai legati] sono acciai per i quali \emph{almeno uno dei valori limite è raggiunto dai dai rispettivi tenori degli elementi in lega} (tabella \ref{tab:Prosp1}) a patto che non siano già appartenenti agli inossidabili.
\end{description}

\begin{table}
\centering
\caption{Prospetto I, norma UNI EN 10020:2001}\label{tab:Prosp1}
\begin{tabularx}{0.5\textwidth}{lXl}
\toprule
\textbf{Elemento} &\textbf{Tenore in $\%$ in massa}\\
\midrule
Al & Alluminio & 0.30\\
B & Boro & 0.0008\\
Bi & Bismuto & 0.10\\
Co & Cobalto & 0.30\\
Cr & Cromo & 0.30\\
Cu & Rame & 0.40\\
La & Lantanidi (singolarmente) & 0.10\\
Mn & Manganese & 1.65\\
Mo & Molibdeno & 0.08\\
Nb & Niobio & 0.06\\
Ni & Nichel & 0.30\\
Pb & Piombo & 0.40\\
Se & Selenio & 0.10\\
Si & Silicio & 0.60\\
Te & Tellurio & 0.10\\
Ti & Titanio & 0.05\\
V & Vanadio & 0.10\\
W & Tungsteno & 0.30\\
Zr & Zircronio & 0.05\\
- & Altri & 0.10\\
\bottomrule
\end{tabularx}
\end{table}

\subsection{Acciai non legati}
\begin{multicols}{2}[]
\subsubsection{Di Qualità}
Sono acciai per i quali, in genere, sussistono prescrizioni riguardanti caratteristiche specifiche, per esempio: tenacità, grossezza e/o formabilità.
Non sono destinati a trattamenti termici (al più a ricottura e normalizzazione).
\columnbreak
\subsubsection{Speciali}
Sono acciai che presentano, rispetto agli acciai non legati di qualità, una maggiore purezza in particolare nei confronti delle inclusioni non metalliche.
In genere presentano risposta regolare ai \ac{TT}, e nella maggior parte dei casi sono destinati a:
\begin{enumerate}
\item trattamento di bonifica,
\item trattamento di tempra superficiale.
\end{enumerate}
Fanno parte di tale classe gli acciai non legati che rispondono a una o più delle seguenti prescrizioni tutte quelle definizioni che rientrano in \ref{sec:ANLS}. \todo{Aggiungere i riferimenti all'apendice}
\end{multicols}

\subsection{Acciai inossidabili}
Sono suddivise in base a due criteri:
\begin{enumerate}
\item tenore di Nichel:
\begin{itemize}
\item Ni $<2.5\%$
\item Ni $>2.5\%$
\end{itemize}
\item caratteristiche paricolari:
\begin{itemize}
\item resistenza alla corrosione;
\item resistenza all'ossidazione a caldo;
\item resistenza allo scorrimento.
\end{itemize}
\end{enumerate}
\newpage
\subsection{Acciai legati}
\begin{multicols}{2}[]
\subsection{di Qualità}
Sono acciai il cui utilizzo è simile agli acciai non legati di qualità, ma che contengono elementi in lega per rispondere ad alcune prescrizioni di impiego.
Non sono, di regola, destinati a trattamento termico di bonifica o ad un tratamento di tempra superficiale.
Ne fanno parte gli acciai definiti in \ref{sec:ALDQ}.
\columnbreak
\subsection{Speciali}
Sono acciai , diversi dagli inossidabili, che non rientrano tra le categorie definite per gli acciai legati di qualità caratterizzati da:
\begin{itemize}
\item regolazione precisa della composizione chimica;
\item particolari condizioni di elaborazione e controllo del processo produttivo.
\end{itemize}
Ne fanno parte gli acciai descritti in \ref{sec:ALS}.
\end{multicols}