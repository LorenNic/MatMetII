\chapter{Classificazione e Designazione degli acciai}\label{chp:ClassAcc}
\section{La normazione}
Per cominciare, è utile osservare come gli enti di normazione descrivono gli acciai.
tra l'altro sono tra i prodotti più normati presenti sul mercato industriale.
Dapprima:
\begin{description}
\item[UNI] sigla che indica una normativa realizzata dall'Ente nazionale di Unificazione.
Ente che norma tutte le attività produttive sul mercato italiano. Inoltre è facente parte del \acs{CEN}. Difatti applica sul suolo italiano tutte le normative date dallo stesso \acs{CEN}.
Non è ammessa la presenza di normative che siano in contrasto con quelle europee.
\item[EN] contraddistingue le norma sviluppate dal \ac{CEN}.
Le normative EN devono essere percepite da tutti gli stati membri dello spazio economico europeo.
Ciò per garantire il libero scambio di prodotti al interno del mercato.
Il EN è composto dai principali enti nazionali di normazione degli stati membri nello spazio economico europeo.
\item[ISO] rappresenta tutte le normative sviluppate dal \ac{ISO}. Possono essere un riferimento applicabile per tutto il mondo. Una nazione può decidere se applicare le norma \acs{ISO} indipendentemente da quanto fatto dal \acs{CEN}.
\end{description} 

Secondo le normative della \acs{CEN} le normative hanno lo scopo di:
\begin{quote}
Stabilire le condizioni tecniche per lo scambio di prodotti e di servizi assicurando il continuo adeguamento allo sviluppo delle tecnologie e dei bisogni del mercato
\end{quote}
con lo scopo di eliminare le barriere commerciali, almeno tra gli stati europei.

Una prima classificazione dei tipi di acciai perché esistono tante classi di materiale.
Dunque si può pensare ad una divisione in base:
\begin{multicols}{2}
\begin{itemize}
\item composizione chimica;
\item processo di fabbricazione;
\item caratteristiche meccanico-fisiche e di impiego;
\columnbreak
\item costituenti strutturali;
\item ecc\dots
\end{itemize}
\end{multicols}

Non a caso sono stati citati i precedenti aspetti, in fatti le normative vanno a coprire gli aspetti stessi, come mostrato nella tabella \ref{tab:NormGen}

\begin{table}
\centering
\caption{Norme di carattere geneale}\label{tab:NormGen}
\begin{tabularx}{\textwidth}{>{\bfseries}lX}
\toprule
UNI EN 10020:2001 & Descrizione e classificazione dei tipi di acciaio\\
UNI EN 10027-1:2016 & Sistemi di designazione degli acciai, \texttt{Designazione alfanumerica}\\
UNI EN 10027-2:2015 & Sistemi di designazione degli acciai, \texttt{Designazione numerica}\\
UNI EN 10025-(1-6):2005 & Prodotti laminati a caldo di acciai per impieghi strutturali\\
UNI EN 10079:2007 & Descrizione dei prodotti di acciaio (forma, dimensioni, aspetto, stato superficiale)\\
\bottomrule
\end{tabularx}
\end{table}

Secondo la norma \texttt{UNI EN 10020:2001}:
\begin{quote}
L'acciaio è un materiale il cui \emph{tenore in massa di Ferro (Fe) è maggiore di quello di ciascuno degli altri elementi} ed il cui \emph{tenore di Carbonio (C) è generalmente minore del $2\%$} e che contiene altri elementi. Un numero limitato di acciai al Cromo (Cr) può avere tenore di carbonio maggiore del $2\%$, ma tale valore del $2\%$ è il tenore limite corrente che separa l'acciaio dalla ghisa.
\end{quote}
Sempre la stessa norma definisce la classificazione principale degli acciai \ref{fig:UNIEN10020:2001}.

\begin{figure}
\usetikzlibrary{trees}
\begin{tikzpicture}[
sibling distance = 10em,
every node/.style={rectangle, rounded corners, draw, align=center,}
]
\node{Acciai}
	child{ node[top color = UnifeLight] {Non legati}
		child{ node[bottom color = UnifeDark!80, white] {di qualità}}
		child{ node[bottom color = UnifeDark!80, white] {speciale}}}
	child{node[top color = UnifeLight] {Inossidabili}}
	child{node[top color = UnifeLight] {Legati}
		child{node[bottom color = UnifeDark!80, white] {di qialità}}
		child{node[bottom color = UnifeDark!80, white] {speciali}}};
\end{tikzpicture}
\caption{Suddivisione acciai in base alla normativa UNI EN 10020:2001}
\label{fig:UNIEN10020:2001}
\end{figure}
\todo{Continuare con il testo}