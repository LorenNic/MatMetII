\chapter{Trattamenti termici}
Granigliatura ha l'obbiettivo di pulire il comportamento.\\
La pallinatura ha come obbiettivo l'incrudimento a freddo del materiale pallinato.

\section{Filiera prodotto acciaio}
Quasi tutti i prodotti vengono da acciaieria in prodotto come barre, in genere. Si ottengono, dal fuso, delle bilette e poi su laminazione a freddo, con riduzione dal 4\% a 8\%.
L'acciaieria presenta un set di diametri con diverse composizioni di acciai.

come mai effettuare una ricottura isotermica sul prodotto forgiato?
Si possono avere delle eccessive deformazioni in caso si vogliano trattare termicamente.

Dopo la vendita allo stato di ricottura isotermica vengono eseguite le lavorazioni meccaniche del caso.
Prima si fanno delle verifiche metallografiche per verificare che il prodotto rispetti i dettagli richiesti.

Poi si eseguono i trattamenti termici.
Quindi si garantisce sia un'ottima durezza superficiale (durezza superficiale) mentre a cuore si preferisce avere una certa tenacità.

In automotive si cercano acciai al massimo $1.10\%$ di tenore di carbonio.
Per esempio solo nel cambio si hanno 18 componenti che in genere vengono cementati e solitamente pallinati.
L'esigenza della ricerca su nuovi \ac{TT} è dovuto al fatto che non è possibile cambiare progetto di una scatola del cambio in continuazione.
Perciò devono aumentare le caratteristiche meccaniche senza le dimensioni.

Segue un processo di pallinatura, dove richiesto.
Ha l'obbiettivo di aumentare la resistenza a fatica.

Nella temprabilità le variabili importanti sono
\begin{itemize}
\item Dimensione grano (è fissato da normativa)
\item Tenore carbonio
\item Tenore di leghe
\end{itemize}

Si possono avere degli acciai microlegati potrebbero essere una valida sostituzione ad acciai da bonifica a patto che il materiale lavori ad una temperatura superiore o uguale a quella ambiente. Altrimenti è obbligatorio l'utilizzo dei bonificati.

Nella cementazione in atmosfera: si formano degli strati di ossidi che compromettono, leggermente compromettenti la resistenza a fatica del materiale cementato.
In caso di bassa pressione si evita la formazione degli ossidi, garantendo una resistenza a fatica di circa il 20\%.
