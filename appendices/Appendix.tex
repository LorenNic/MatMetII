%!TEX root = ../MatMetII.tex

%==============================================================================================================================================
\chapter{Considerazioni aggiuntive sulla UNI EN 10020}\label{app:sus}
\section{Tipologie di acciai non legati speciali}\label{sec:ANLS}
Di seguito sono riportati quali acciai rientrano in questa classe. 
\begin{enumerate}
\item acciai che presentano un valore minimo di resilienza allo stato bonificato;
\item acciai che presentano un valore stabilito di profondità di penetrazione di tempra o di durezza superficiale allo stato temprato, bonificato o indurito superficialmente.
\item acciai per i quali sono prescritti tenori particolarmente ridotti di inclusioni non metalliche.
\item acciai con tenore massimo di S e P $\leq 0.020\%$ su analisi di colata.
\item resilienza $\geq 27\unit{\J}$ a $-50\unit{\degree}$ su provini Charpy a V in senso longitudinale.
\item acciai per reattori nucleari con limitazioni su tenori di Cu $\leq 0.10\%$, Co $\leq 0.05\%$ e V $\leq 0.05\%$.
\item acciai che presentano conduttività elettrica $\geq 9\unit{Sm/\mm^2}$.
\item acciai per cemento armato precompresso.
\item acciai indurenti per precipitazione con C $>0.25\%$ con struttura di ferrite-perlite, con aggiunta di micro-leganti come Nb e V (sotto ai limiti del prospetto \ref{tab:Prosp1}).
\end{enumerate}

\section{Tipologie di acciai legati di qualità}\label{sec:ALDQ}
\begin{enumerate}
\item Acciai saldabili a grano fine per impieghi strutturali, che rispondano contemporaneamente alle seguenti prescrizioni:
	\begin{itemize}
	\item $R_{s,min} < 380\unit{\MPa}$ $(s < 16\unit{\mm})$;
	\item valore degli elementi in lega inferiori a valori imposti rigorosamente dalla norma;
	\item acciai con valore minimo di KV $\leq 27\unit{\J}$ (provetta Charpy, intaglio a V, $-50\unit{\degree}$).
	\end{itemize}
\item acciai che contengono solo Si (o Si e Al) come elementi in lega, con prescrizioni riguardanti la limitazione delle perdite magnetiche e/o dei valori minimi dell'induzione magnetica;
\item acciai per rotaie, per parancole e armature di miniere;
\item acciai legati per i quali il Cu è il solo elemento prescritto;
\item acciai legati per prodotti piani laminati a caldo o a freddo destinati a operazioni severe di deformazioni a freddo e contenenti elementi affinanti il grano quali B, Nb, Ti, V e/o Zr;
\item acciai bifasici
\end{enumerate}

\section{Tipologie di acciai legati speciali}\label{sec:ALS}
\begin{enumerate}
\item per costruzioni meccaniche, per apparecchi a pressione, e/o con caratteristiche fisiche particolari;
\item acciai rapidi, acciai da utensili;
\item acciai per cuscinetti e altri acciai per usi particolari;
\end{enumerate}

%%%%%%%%%%%%%%%%%%%%%%%%%%%%%%%%%%%%%%%%%%%%%%%%%%%%%%%%%%%%%%%%%%%%%%%%%%%%%%%%%%%%%%%%%%%%%%%
%%%%%%%%%%%%%%%%%%%%%%%%%%%%%%%%%%%%%%%%%%%%%%%%%%%%%%%%%%%%%%%%%%%%%%%%%%%%%%%%%%%%%%%%%%%%%%%
%%%%%%%%%%%%%%%%%%%%%%%%%%%%%%%%%%%%%%%%%%%%%%%%%%%%%%%%%%%%%%%%%%%%%%%%%%%%%%%%%%%%%%%%%%%%%%%

\chapter{Considerazioni aggiuntive sugli acciai da costruzione}

\section{Accia effervescenti e calmati}\label{sc:AccEffCalm}
Durante la colata, ci possono essere dello formazioni di ossidi di carbonio per via dell'alta 
reattività tra ossigeno e carbonio appunto.
Ciò provoca la presenza di impurità decisamente grandi al interno del materiale colato.
Per eliminare la probabilità di tali formazioni, che ovviamente intaccano le prestazioni 
meccaniche dell'acciaio, si inseriscono in colata degli elementi ad alta reattività con 
l'ossigeno per poterlo estrarre in fase di raffreddamento del colato.
Tali sono:
\begin{description}
\item[Acciai effervescenti] acciai a cui, durante la colata, \textbf{non} vengono aggiunti elementi. Per cui presentano dispersione di $CO$ al interno, sotto forma di bolle - da cui il nome effervescente. Il vantaggio è il costo decisamente inferiore rispetto ai successivi e non presentano ritrazione in fase di raffreddamento.
\item[Acciai calmati] acciai a cui vengono inseriti durante la colata elementi ad alta reattività con l'ossigeno, tipo $Al$, $Si$ o $Mn$. I quali legano l'ossigeno e lo portano in superficie per via della minore densità. Da ciò si forma una schiuma che può essere eliminata agilmente prima del raffreddamento del colato. Grazie a questa tecnica si ha un maggiore controllo sul tenore del carbonio presente in lega. 
\end{description}
Al giorno d'oggi, gli acciai venduti sono tutti calmati. È imposto solo per alcune normative: dunque gli acciai effervescenti si possono ancora trovare sul mercato. Il punto è che il loro utilizzo è molto limitato per via della poca affidabilità nelle caratteristiche meccaniche.

\section{Considerazioni sugli HSLA}\label{sc:HSLACons}
Lo sviluppo di questi nuovi acciai testimonia come la ricerca può portare
a risultati di interesse applicativo.
Risulta fondamentale l'acquisizione di conoscenze sul rapporto tra
struttura e proprietà.
Lo sviluppo di nuove metodologie di analisi, hanno consentito una 
migliore definizione e indagine delle strutture.