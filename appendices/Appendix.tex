%!TEX root = ../Report.tex

%==============================================================================================================================================
\chapter{Considerazioni aggiuntive sulla UNI EN 10020}\label{app:sus}
\section{Tipologie di acciai non legati speciali}\label{sec:ANLS}
Di seguito sono riportati quali acciai rientrano in questa classe. 
\begin{enumerate}
\item acciai che presentano un valore minimo di resilienza allo stato bonificato;
\item acciai che presentano un valore stabilito di profondità di penetrazione di tempra o di durezza superficiale allo stato temprato, bonificato o indurito superficialmente.
\item acciai per i quali sono prescritti tenori particolarmente ridotti di inclusioni non metalliche.
\item acciai con tenore massimo di S e P $\leq 0.020\%$ su analisi di colata.
\item resilienza $\geq 27\unit{\J}$ a $-50\unit{\degree}$ su provini Charpy a V in senso longitudinale.
\item acciai per reattori nucleari con limitazioni su tenori di Cu $\leq 0.10\%$, Co $\leq 0.05\%$ e V $\leq 0.05\%$.
\item acciai che presentano conduttività elettrica $\geq 9\unit{Sm/\mm^2}$.
\item acciai per cemento armato precompresso.
\item acciai indurenti per precipitazione con C $>0.25\%$ con struttura di ferrite-perlite, con aggiunta di micro-leganti come Nb e V (sotto ai limiti del prospetto \ref{tab:Prosp1}).
\end{enumerate}

\section{Tipologie di acciai legati di qualità}\label{sec:ALDQ}
\begin{enumerate}
\item Acciai saldabili a grano fine per impieghi strutturali, che rispondano contemporaneamente alle seguenti prescrizioni:
	\begin{itemize}
	\item $R_{s,min} < 380\unit{\MPa}$ $(s < 16\unit{\mm})$;
	\item valore degli elementi in lega inferiori a valori imposti rigorosamente dalla norma;
	\item acciai con valore minimo di KV $\leq 27\unit{\J}$ (provetta Charpy, intaglio a V, $-50\unit{\degree}$).
	\end{itemize}
\item acciai che contengono solo Si (o Si e Al) come elementi in lega, con prescrizioni riguardanti la limitazione delle perdite magnetiche e/o dei valori minimi dell'induzione magnetica;
\item acciai per rotaie, per parancole e armature di miniere;
\item acciai legati per i quali il Cu è il solo elemento prescritto;
\item acciai legati per prodotti piani laminati a caldo o a freddo destinati a operazioni severe di deformazioni a freddo e contenenti elementi affinanti il grano quali B, Nb, Ti, V e/o Zr;
\item acciai bifasici
\end{enumerate}

\section{Tipologie di acciai legati speciali}\label{sec:ALS}
\begin{enumerate}
\item per costruzioni meccaniche, per apparecchi a pressione, e/o con caratteristiche fisiche particolari;
\item acciai rapidi, acciai da utensili;
\item acciai per cuscinetti e altri acciai per usi particolari;
\end{enumerate}