\chapter{High Entropy Alloys}\label{chp:HEA}
Sono delle leghe a matrice con più di 5 elementi precipitanti.
Due to the distinct design concept, these presents unusual properties.
These alloys have named HEAs becaouse of their liquid or random solid solutions states have significant\todo{\\ADD}

\missingfigure{Papers production on HEA}
\missingfigure{Mixture of ctristal structure}
As we can see we want to mix these elements to catch al the charateristics produce by singular elements.
This is done by the presence of an high number of alloy in the material.

\begin{quote}
Why are we looking to produce that type of alloys?
\end{quote}

It's becouse they have an high creep reistance on high temperatures. So new application can be developed using these alloys.
Not of all these alloys are temperature hardening. It depends on the singular alloy.
Also higher yeld stress resistance can be achieved by these alloys. As before it depends in single material.
Some HEA have been tested to corrosion. 
In addition, some oh them have high ware resistance.
\missingfigure{Specific stength on Elastic module}
Fracture toughtness is really high even than standards Steel alloys.
So it's been evident how these alloys could find critical application witch are not have impemented yet.

the production is a bit complex.
It's made by sinterization. It's possible to melt them, in particoular induction melting ist one of the most used way to produce HEAs.
Thin films are in develpment as a production process. 

Atomization is a key process, not only on HEAs production, but in general for addictive manufacturing.
So it's foundamental that cooling must be as fast as possible in order to produce thin powder.
After that the milling process is succede. The problem is that in standard milling is done in about $50\div 100 \unit{rpm}$ but for HEAs its going to be about $100 \div 1000\unit{rpm}$. At this moment it's possibile only at laboratory level and not in industrial.

